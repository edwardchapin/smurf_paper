%------------------------------------------------------------------------------
% SMURF Paper
%------------------------------------------------------------------------------

\documentclass[useAMS,usenatbib,usegraphicx,nofootinbib]{mn2e}

\usepackage{amsmath}
\usepackage{url}
\usepackage{natbib}
\usepackage{rotating}

% --- Some user defined macros ------------------------------------------------
\newcommand{\apj}{\rm ApJ}
\newcommand{\apjl}{\rm ApJL}
\newcommand{\apjs}{\rm ApJS}
\newcommand{\aaps}{\rm A$\&$AS}
\newcommand{\aap}{\rm A$\&$A}
\newcommand{\aapr}{\rm A$\&$AR}
\newcommand{\mnras}{\rm MNRAS}
\newcommand{\aj}{\rm Astron. J.}
\newcommand{\araa}{\rm ARAA}
\newcommand{\nat}{\rm Nature}
\newcommand{\pasj}{\rm PASJ}
\newcommand{\pasp}{\rm Publ. Astron. Soc. Pac.}
\newcommand{\ASP}{\rm ASP COnference Series}
\newcommand{\CASP}{\rm Comm. Astrophys. Space Phys.}
\newcommand{\astroph}{\rm astro-ph/}

\newcommand{\snr}{SNR}

\newcommand{\scuba}{SCUBA-2}

\def\lsim{\mathrel{\lower2.5pt\vbox{\lineskip=0pt\baselineskip=0pt
           \hbox{$<$}\hbox{$\sim$}}}}

 \def\gsim{\mathrel{\lower2.5pt\vbox{\lineskip=0pt\baselineskip=0pt
           \hbox{$>$}\hbox{$\sim$}}}}


% ----------------------------------------------------------------------------

\title[SMURF: an iterative map-maker for SCUBA-2]{The Sub-Millimetre User
Reduction Facility: an iterative map-maker for SCUBA-2}

\author[Edward~L.~Chapin~et~al.]{
  \parbox[t]{\textwidth}{
    Edward~L.~Chapin$^{1}$\thanks{E-mail:~echapin@phas.ubc.ca},
    David~S.~Berry$^{2}$,
    Andrew~G.~Gibb$^{1}$,
    Tim~Jenness$^{2}$,
    Douglas~Scott$^{1}$
  }
  \\
  \\
  $^{1}$Dept. of Physics \& Astronomy, University of British Columbia,
  6224 Agricultural Road, Vancouver, B.C. V6T 1Z1, Canada\\
  $^{2}$JointAstronomy Centre, 660 N. A‘oh ̄ k ̄ Place, University Park, Hilo, Hawaii 96720, USA}

\begin{document}

\label{firstpage}

\maketitle

\begin{abstract}
  We describe the Sub-Millimetre User Reduction Facility (SMURF), an
  iterative map-maker that was developed for the Submillimetre Common
  User Bolometer Array 2 (SCUBA-2).
\end{abstract}


\begin{keywords}
bla, bla, bla
\end{keywords}

%------------------------------------------------------------------------------
\section{Introduction}
\label{sec:intro}
%------------------------------------------------------------------------------

The Submillimetre Common User Bolometer Array 2
\citep[\scuba,][]{holland2006} is a new instrument for the the 15-m
James Clerk Maxwell Telescope (JCMT) on Mauna Kea, Hawai'i. The camera
can simultaneously image the sky in two broad bands centered over 450
and 850\,\micron. Ultimately, the instrument will also be equipped
with a fourier transform spectrometer (FTS), and a polarimeter
(POL). This paper describes the Submillimetre User Reduction Facility,
SMURF, a software package for reducing the imaging data, with an
emphasis on data taken during the SCUBA-2 Shared Risk Observing period
(S2SRO) which took place from February to April of 2010. The reduction
of FTS and POL data will be described at a later date once these
additional instruments have been commissioned.

Over the last twenty years, the submillimetre band (defined here to be
200--1000\,\micron) has revolutionized several important areas of
astrophysics: helping to characterize the early stages of
star-formation by identifying the dense, cold regions in molecular
clouds where stars may eventually form; discovering through blind
surveys the locations and surface density of a class of massive
star-forming galaxies in the early Universe, now referred to as
submillimetre galaxies, or SMGs; and finding debris disks around
nearby stars, the early stages of planet formation.

The most common source of emission at these wavelengths in the
Universe is the thermal emission from cool dust.


Styles of map-making: maximum likelihood
\citep[e.g.,][]{patanchon2008}; iterative -- see \citet{johnstone2000}
implementation of \citet{wright1996} to remove chop,
\citet{kovacs2008} for SHARC-2 etc.; de-correlation using PCA
analysis \citep[e.g.][]{laurent2005,scott2008,aguirre2010}.

Since \scuba\ will ultimately have nearly 10,000 working bolometers,
considerably larger than any other existing ground, or space-spaced
bolometer cameras, SMURF has been designed to use a less accurate,
though faster and less memory-intensive iterative approach that
attempts to model and remove most of the correlated signal components,
and then regrid the residuals signals. This method is closest in
spirit to the Comprehensive Reduction Utility for SHARC-2
\citep[CRUSH,][]{kovacs2008}.


%------------------------------------------------------------------------------
\section{Data Properties}
\label{sec:data}
%------------------------------------------------------------------------------

\subsection{How \scuba\ takes measurements}

A brief summary of the instrument hitting on topics that will be
discussed later: TES operation; base temperature and pixel heaters to
compensate for sky variations; measuring flatfields; sample rate; scan
speed to move astronomical sources of interest to higher frequencies
in the time series.

\subsection{\scuba\ shared-risk observations}

This paper will be focussed on S2SRO: only one subarray at each
wavelength; fridge oscillations; working detectors.

Show some time series.

Show power spectra.

Show cross-spectra to emphasize how correlated things are?

%------------------------------------------------------------------------------
\section{The SMURF Algorithm}
\label{sec:data}
%------------------------------------------------------------------------------

Describe the bolometer signal model, and with the help of a flow-chart
show the basic procedure that SMURF follows.

We express the signal observed by the $i$th bolometer as a function of time,
\begin{equation}
B_i(t) = E(t)[ A(t) + N(t) ] + C(t)
\end{equation}
where $A(t)$ is the time-varying astronomical signal produced by
scanning the telescope across sources, $C(t)$ is a common signal
observed by all of the detectors, $E(t)$ is the time-varying
extinction, which is a function of the telescope elevation and
observing conditions, and $N(t)$ represents all other noise terms. The
common-mode, $C(t)$ has a strong 

\subsection{Data cleaning}

\subsubsection{Step correction}

\subsubsection{Flatfields using heater ramps}

\subsubsection{Flatfields using the common-mode signal}



\subsection{Correlated low-frequency noise}
At the end of the day, we iteratively fit and remove the common-mode
signal, and then any residual noise is removed with an iterative
high-pass filter. Here are the components that we know about:

\subsubsection{Fridge oscillations}
Talk about the fridge oscillation and how there might be residuals
correlated with things like the pattern of G or Tc.

\subsubsection{Sky noise}
Do we even see sky noise? Can show some results for the combined 450
and 850\,\micron\ joint PCA analysis.

\subsubsection{Magnetic field pickup}
Show that it occasionally appears in the dark squids.




\subsection{Noise estimation}

Measure a white noise level from the power spectrum for each detector
to use as a weight. In the map, use weighted standard error to
estimate noise. This is accurate for a single pixel, but there is
still spatially correlated noise in the maps which we can show with
some figures.


\subsection{Model degeneracies}

Introduce the topic of model degeneracies here, but defer their
solution to the example sections.


%------------------------------------------------------------------------------
\section{Examples}
\label{sec:examples}
%------------------------------------------------------------------------------

\subsection{Point source}
Reduce a calibrator. Describe the use of a zero mask around the
source. Compare with \citet{wiebe2009} who used the same method to
constrain a maximum-likelihood map with poor cross-linking.

\subsection{Deep blind survey}
Heavy initial filtering, and only iteratively solve for map and
common-mode with edge constraints.


\subsection{Bright extended emission}
The hardest thing to do with SCUBA-2 at the moment. Try to generate a
source/background mask to assist the map solution based on thresholds
in \snr.

%------------------------------------------------------------------------------
\section{Summary and future work}
\label{sec:summary}
%------------------------------------------------------------------------------

\scuba\ is being re-commissioned. Maybe we will have some initial data
on fridge performance? If both it, and sky noise are relatively
well-behaved, might be able to restrict filtering to lower frequencies
and get larger-scale structures.

Larger array footprint also helps.


%------------------------------------------------------------------------------
\section{Acknowledgements}
%------------------------------------------------------------------------------


%------------------------------------------------------------------------------
\bibliographystyle{mn2e}
\bibliography{mn-jour,refs}
%------------------------------------------------------------------------------

%------------------------------------------------------------------------------
\appendix
\section[]{This is an appendix}
%------------------------------------------------------------------------------


\end{document}
