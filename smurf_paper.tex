%------------------------------------------------------------------------------
% SMURF Paper
%------------------------------------------------------------------------------

\documentclass[useAMS,usenatbib,nofootinbib]{mn2e}

% --- Uncomment following block for use with pdflatex and pdf images ----------
%% Recommended by MN for handling hyperlinks

\usepackage[pdftex,pdfpagemode={UseOutlines},bookmarks,bookmarksopen,colorlinks,linkcolor={blue},citecolor={green},urlcolor={red}]{hyperref}

\usepackage[pdftex]{graphicx}

% add a .pdf suffix to image file names
\newcommand{\imagefile}[1]{#1.pdf}
% -----------------------------------------------------------------------------



% --- Uncomment following block for latex and postscript images ---------------
%% Recommended by MN for handling hyperlinks - PS version

%\usepackage[bookmarks,bookmarksopen,colorlinks,linkcolor={blue},citecolor={green},urlcolor={red}]{hyperref}

% add a .eps suffix to image file names
%\newcommand{\imagefile}[1]{#1.eps}
% -----------------------------------------------------------------------------



\usepackage{amsmath}
\usepackage{url}
\usepackage{natbib}
\usepackage{rotating}

% --- Some user defined macros ------------------------------------------------

% journals
\newcommand{\ao}{AO}
\newcommand{\apj}{ApJ}
\newcommand{\apjl}{ApJL}
\newcommand{\apjs}{ApJS}
\newcommand{\aaps}{A$\&$AS}
\newcommand{\aap}{A$\&$A}
\newcommand{\aapr}{A$\&$AR}
\newcommand{\mnras}{MNRAS}
\newcommand{\aj}{A J}
\newcommand{\araa}{\rm ARA\&A}
\newcommand{\nat}{Nat}
\newcommand{\pasj}{PASJ}
\newcommand{\pasp}{PASP}
\newcommand{\prd}{PhRvD}
\newcommand{\ASP}{ASP Conf.\ Ser.~}
\newcommand{\CASP}{\rm Comm. Astrophys. Space Phys.}
\newcommand{\astroph}{\rm astro-ph/}
\newcommand{\spie}{Proc.\ SPIE~}


% common acronyms etc.
\newcommand{\snr}{SNR}
\newcommand{\scuba}{SCUBA-2}
\newcommand{\rms}{RMS}

% these are approximately less-than and greater-than symbols
\def\lsim{\mathrel{\lower2.5pt\vbox{\lineskip=0pt\baselineskip=0pt
          \hbox{$<$}\hbox{$\sim$}}}}

\def\gsim{\mathrel{\lower2.5pt\vbox{\lineskip=0pt\baselineskip=0pt
          \hbox{$>$}\hbox{$\sim$}}}}

% sinc function
\def\sinc{\mathrm{sinc}}

% how are model names displayed
\newcommand{\model}[1]{\texttt{#1}}


% ----------------------------------------------------------------------------

\title[\scuba: iterative map-making with SMURF]{\scuba: iterative map-making
with the Sub-Millimetre User Reduction Facility}

\author[Edward~L.~Chapin~et~al.]{
  \parbox[t]{\textwidth}{
    Edward~L.~Chapin$^{1,2}$\thanks{E-mail:~echapin@sciops.esa.int,
      Present address: XMM SOC, ESAC, Apartado 78, 28691 Villanueva de
      la Ca\~nada, Madrid, Spain},
    David~S.~Berry$^{2}$,
    Andrew~G.~Gibb$^{1}$,
    Tim~Jenness$^{2}$,
    Douglas~Scott$^{1}$,
    Remo~P.~J.~Tilanus$^{2,3}$,
    Frossie~Economou$^2$\thanks{Present address: National Optical
      Astronomy Observatory, 950 N.\ Cherry Avenue, Tucson, AZ 85719,
      USA},
    Wayne ~S.~Holland$^{4,5}$
  }
  \\
  \\
  $^{1}$Dept. of Physics \& Astronomy, University of British Columbia,
  6224 Agricultural Road, Vancouver, B.C. V6T 1Z1, Canada\\
  $^{2}$JointAstronomy Centre, 660 N. A`oh\={o}k\={u} Place, University
  Park, Hilo, Hawaii 96720, USA\\
  $^{3}$Netherlands Organisation for Scientific Research,
  Laan van Nieuw Oost-Indie 300, NL-2509 AC The Hague, The Netherlands\\
  $^{4}$UK Astronomy Technology Centre, Royal Observatory, Blackford
  Hill, Edinburgh EH9 3HJ\\
  $^{5}$Institute for Astronomy, University of Edinburgh, Royal Observatory, Blackford Hill Edinburgh, EH9 3HJ}
\begin{document}

\label{firstpage}

\maketitle

\begin{abstract}
  The Submillimetre Common User Bolometer Array 2 (\scuba) is an
  instrument operating on the 15-m James Clerk Maxwell Telescope,
  consisting of about 5000 bolometers in each of two simultaneous
  imaging bands centred over 450 and 850\,\micron. The camera is
  operated by scanning across the sky and records data at a rate of
  200\,Hz. This rate is much greater than that of previously existing
  submillimetre cameras, and represents a significant analysis
  challenge. We describe the production of \scuba\ maps using the
  Sub-Millimetre User Reduction Facility (SMURF) in which we have
  adopted a fast, iterative approach to map-making that enables data
  reduction on single, modern, high-end desktop computers, with
  execution times that are shorter than the observing times.  SMURF is
  used in an automated setting, both at the telescope for real-time
  feedback to observers, as well as for the production of generic
  science products for the JCMT Science Archive at the Canadian
  Astronomy Data Centre. Three detailed case studies are used to: (i)
  explore convergence properties of the map-maker using simple prior
  constraints (Uranus -- a point source); (ii) achieve the white-noise
  limit for faint point-source studies (extragalactic blank-field
  survey of the Lockman Hole); and (iii) demonstrate that our strategy
  is capable of recovering angular scales comparable to the size of
  the array footprint (approximately 5\,arcmin) for bright extended
  sources (star-forming region M17).
\end{abstract}


\begin{keywords}
methods: data analysis, techniques: image processing, submillimetre:
general, methods: observational
\end{keywords}

%------------------------------------------------------------------------------
\section{Introduction}
\label{sec:intro}
%------------------------------------------------------------------------------

The Submillimetre Common User Bolometer Array 2
\citep[\scuba,][]{holland2012} is a new instrument for the 15-m James
Clerk Maxwell Telescope (JCMT) on Mauna Kea, Hawai'i. The camera
simultaneously images the sky in two broad bands centred over 450 and
850\,\micron, with approximately 7.5 and 14.5\,arcsec full-width at
half-maximum (FWHM) point spread functions (PSFs). The focal planes at
each wavelength are populated with 4 rectangular subarrays, consisting
of $40 \times 32$ bolometers each, and together subtend a nearly
7\,arcmin $\times$ 7\,arcmin footprint on the sky (excluding gaps, the
continuous solid angle is about 43\,arcmin$^{2}$ per focal plane).
This paper describes the properties of \scuba\ data that are relevant
for producing maps of the imaging data, and the Submillimetre User
Reduction Facility, SMURF, a software package for performing the
reduction written using the Starlink Software Environment
\citep{1993ASPC...52..229W,2009ASPC..411..418J}. The details of the instrument design,
performance, and calibration are given in two companion papers:
\citet{holland2012} and \citet{dempsey2012}.

Over the last twenty years, observations in the submillimetre band
(defined here to be 200--1200\,\micron) have helped revolutionise
several important areas of astrophysics, including: discovering
through blind surveys a class of massive dusty star-forming galaxies
in the early ($z>2$) Universe, now referred to as submillimetre
galaxies, or SMGs; the characterisation of the early stages of
star-formation by identifying the dense, cold regions in molecular
clouds where stars may eventually form; and identifying debris disks
around nearby stars, helping us understand the early stages of planet
formation.  With $\sim10,000$ total detectors, \scuba\ is presently
the largest submillimetre camera in the world, and the fastest
wide-area ground-based submillimetre imager.

Submillimetre imaging cameras generally use bolometers, rather than
coherent detectors, to maximise sensitivity. The sensitivity of
bolometers is limited by white photon and phonon noise from the
instrument and ambient backgrounds. The low-frequency noise, however,
is typically dominated by sources which produce slow variations in the
background (e.g., thermal variations within the cryostat, and the
atmosphere for ground-based cameras), and drifts in the readout
electronics. Such noise has a power spectrum $\propto 1/f^\alpha$
($\alpha>0$), and the frequency at which it is comparable to the white
noise level is called the ``$1/f$ knee''. Since the low-frequency
noise is largely correlated between all, or subsets, of the bolometers
in time, it can be suppressed during map-making, since astronomical
signals have the distinct property that they are fixed in a sky
reference frame (assuming they are not time-varying). If successful,
the noise in the resulting map is said to be ``white noise limited'',
meaning that it is uncorrelated spatially, and has an amplitude that
scales as the $\mathrm{NEFD}/\sqrt{t}$, where the NEFD is the
noise-equivalent flux-density (the white noise level of a bolometer in
1\,s of integration), and $t$ is the amount of integration time in a
map pixel.

There are numerous ways to attack this map-making problem, both in
terms of the data-collection method, and processing. The two most
important principles to follow in terms of scan strategy are: (i) to
modulate the astronomical signals of interest in such a way that they
appear in the lowest-noise regions of the bolometer noise power
spectrum, i.e., above the $1/f$ knee; and (ii) to provide good
``cross-linking'', in which each portion of the map is sampled on a
range of temporal scales, again, to help distinguish time-varying
noise features from fixed astronomical sources. In the case of \scuba,
(i) is achieved through fast-scanning of the entire telescope (up to
600\,arcsec\,sec$^{-1}$), such that significant drift in the
bolometers due to low-frequency noise occurs more slowly than the
crossing times for the astronomical scales of interest; and (ii) by
offering scan patterns that cross the sky at a wide range of position
angles, ultimately observing each portion of the map at a number of
different times. Such methods are now used by virtually all existing
ground-based submillimetre cameras
\citep[e.g.,][]{glenn1998,weferling2002,wilson2008,kovacs2008b}, in
preference to ``chopping'' methods (where the secondary is moved
quickly to modulate the signal) that were more appropriate for older
instruments that had poorer low-frequency noise performance, and were
only sensitive to modest angular scales

There are three general styles of map-making that are relevant to
reducing bolometer data in the literature. The simplest ``direct
methods'' involve some basic processing of the data to remove as much
noise contamination as possible, (e.g., using baseline removal and
other simple filters), and then re-gridding these cleaned data into a
map. Such was the basic recipe for the reduction of chopped data from
SCUBA-2's predecessor SCUBA
\citep{1998ASPC..145..216J,2000ASPC..216..559J}, and MAMBO
\citep[e.g.,][]{omont2001}, another camera from the same
generation. Generally speaking, such methods are fast, although
depending on the science goals and noise properties of the data, they
may not achieve the best noise performance on the angular scales of
interest. In more recent years a method that has been popular for
reducing data in fields of faint point sources with Bolocam
\citep[e.g.,][]{laurent2005}, and its younger sibling the Aztronomical
Thermal Emission Camera \citep[AzTEC, e.g.,][]{scott2008}, is
``principal component analysis'' (PCA) cleaning. A statistical
``black-box'' removes the most correlated components of the bolometer
signals, enabling the detection of point-sources very close to the
theoretical white-noise limits of the detectors, with reasonable
computation times when small numbers of bolometers are involved
(hundreds rather than thousands of detectors). However, PCA cleaning
is not a good generic solution for producing maps of extended
structures, since such sources produce correlated signals amongst many
detectors, and are removed by this procedure. Furthermore, in the case
of \scuba\, performing PCA on even a single subarray (1280 detectors,
nominally) can be prohibitively slow.

The best existing map-making strategies for recovering information on
all angular scales are maximum likelihood techniques, in which the
time-series data are expressed as a sampling of the ``true'' map of
the sky plus noise, and then the equation is inverted to estimate the
map as some weighted linear combination of the data that minimises the
variance.  The first good description of this method appears in
\citet{janssen1992} for the production of maps from the COsmic
Background Explorer (COBE -- the description is relevant despite the
fact that it used a differential radiometer instead of
bolometers). Other descriptions in the experimental cosmic microwave
background (CMB) literature include \citet{tegmark1997} and
\citet{stompor2002}, and an application to data from SCUBA is
described in \citet{borys2004}.  The downside to such methods is that
they can be both computationally expensive and have large memory
requirements. While for some experimental designs fast iterative
methods for the inversion do exist without requiring excessive amounts
of memory \citep[e.g.,][]{wright1996}, for the more general map-making
problem, involving many detectors, things are significantly
complicated both by the need to measure the cross power spectra for
all unique pairs of bolometers, as well as performing the inversion
itself. The most promising maximum-likelihood method that may one day
be applied to \scuba\ data is ``SANEPIC'', which successfully reduced
maps from the Balloon-borne Large-Aperture Submillimeter Telescope,
involving hundreds of bolometers, while correctly incorporating
inter-bolometer noise correlations \citep[][]{patanchon2008}.

The third approach adopted here for \scuba\ is a compromise between
the previous two methods. Under the assumption that a significant
portion of the (predominantly low-frequency) non-white noise sources
can be modelled, an iterative solution is obtained for both the
astronomical image and the parameters of the noise model. Since the
remaining (non-modelled) noise sources are assumed to be white, a
single scalar \rms\ may be calculated for all of the data points from
a given bolometer to characterise its noise distribution (since the
noise at any instant in time is uncorrelated with others, and with the
data from other bolometers), greatly simplifying the measurement of
noise properties and the inversion step that is so complicated in the
maximum-likelihood methods. In principle, it should also be possible
to recover a wider dynamic range in angular scales than possible for
the simplest direct methods, with an increase in calculation time that
is merely linear in the number of iterations needed, while using a
comparable amount of memory. The iterative method of
\citet{wright1996} specific to chopped (CMB) data was applied to SCUBA
observations by \citet{johnstone2000}, although in this case the
chopping removes most of the low-frequency noise, and greatly
simplifies the problem (at the expensive of large-scale structure in
the data). More similar to the algorithm described here was the
iterative pipeline recipe for fitting and removing baseline drifts in
SCUBA scan-map data as the astronomical image estimate improved
\citep{1999ASPC..172..171J}. The closest modern relatives of SMURF are
the Comprehensive Reduction Utility \citep[CRUSH,][]{kovacs2008} for
the Submillimeter High-Angular Resolution Camera 2 (SHARC-2), the
Bolometer array data Analysis software (BoA) for the Large APEX
BOlometer CAmera \citep[LABOCA,][]{siringo2009}, and the pipeline
developed for the Bolocam Galactic Plane Survey \citep[][which uses
iterative PCA cleaning to avoid the suppression of large-scale
structure]{aguirre2011}.

A reasonable model for correlated noise in \scuba\ is a single
``common-mode'' signal seen by all of the bolometers. Iterative
estimation and removal of this signal significantly lowers the $1/f$
knee, without compromising structures on angular scales smaller than
the array footprint. Residual independent drifts at lower frequencies
are removed with an iterative high-pass filter. This strategy enables
SMURF to reduce data faster than they are taken, on single high-end
desktop computers, and it has successfully been used as part of a
real-time pipeline offering feedback to observers at the
telescope. The pipeline is also used to generate near-science grade
products for the JCMT Science Archive \citep{2011ASPC..442..203E}
hosted by the Canadian Astronomy Data Centre (CADC). Details of the
pipeline design are given in \citet{gibb2005} and
\citet{2008AN....329..295C}.

This paper is organised as follows. We first describe the properties
of \scuba\ data, including a principal component analysis to reveal
correlated noise features, in Section~\ref{sec:data}. Next, the
details of the SMURF algorithm (pre-processing steps and the iterative
solution) are given in Section~\ref{sec:algorithm}. The paper is
concluded in Section~\ref{sec:examples} with three detailed test cases
that span the majority of observation types likely to be undertaken
with \scuba, with an emphasis on the mitigation of divergence
problems, and characterising the output maps: (i) Uranus, a bright,
compact source (Section~\ref{sec:point}); (ii) the Lockman Hole, a
blind survey of faint point-like sources (Section~\ref{sec:cosmo});
and (iii) the star-forming region M17, including bright, extended
emission (Section~\ref{sec:extended}). All of the data analysed in
this paper are publicly available through the CADC \scuba\ raw-data
queries page for the dates and observation numbers given in the
text\footnote{\url{http://www.cadc-ccda.hia-iha.nrc-cnrc.gc.ca/jcmt/search/scuba2}}.
All of the analysis was performed using the Starlink \textsc{kapuahi}
release from 2012.

%------------------------------------------------------------------------------
\section{\scuba\ data properties}
\label{sec:data}
%------------------------------------------------------------------------------

In this section we summarise how \scuba\ works,
(Section~\ref{sec:bolos}), give examples of typical bolometers signals
(Section~\ref{sec:bolosignal}), and the impact of magnetic field
pickup (Section~\ref{sec:magpickup}), and finally illustrate the use
of principal component analysis to explore the correlated noise
properties of \scuba\ data (Section~\ref{sec:pca}).

%--------------------------------------------------
\subsection{Description of \scuba}
\label{sec:bolos}
%--------------------------------------------------

While the details of how \scuba\ works are described in
\citet{holland2012}, and its calibration in \citet{dempsey2012}, we
summarise the basic concepts relevant to map-making here.

Incoming light passes through a beam-splitter, and then bandpass
filters, providing simultaneous illumination and wavelength definition
of both the 450 and 850\,\micron\ focal planes. Each focal plan is
populated by four ``subarrays'' (labeled s4a--s4d at 450\,\micron, and
s8a--s8d at 850\,\micron), each consisting of 32 columns and 40 rows
of bolometers. The bolometers are thermal absorbers coupled to
superconducting transition-edge sensors (TESs) for thermometry.
Temperature variations in the TESs produce changing currents, and
therefore varying magnetic fields, which are detected and amplified
using chains of superconducting quantum interference devices (SQUIDs),
before the larger output currents are digitised. Each detector has its
own SQUID for the first-stage of the amplification, but the remaining
stages occur within a common chain of SQUIDs for each of the 32
columns. All 40 rows are read out in sequence at a row-visit frequency
of about 12\,kHz. Such a high sample rate is unnecessary to produce
maps, so the data are re-sampled to approximately 200\,Hz before
writing to disk. This rate provides a sample every 3.0\,arcsec, or
approximately one third of the 450\,\micron\ diffraction-limited
full-width at half-maximum (FWHM) -- a typical rule-of-thumb for
adequately sampling a Gaussian point spread function (PSF) -- at the maximum scanning speed. There is
an additional 41st row of SQUID readouts that are not connected to
TESs. These ``dark SQUIDs'' track non-thermal noise sources that are
common to each column's amplifier chain. The relationship between the
output digitised current, $I$, and the input power, $P$, or $dI/dP$ is
established using flatfield observations immediately prior to science
observations, in which the output signal is measured throughout a ramp
of the pixel heaters (which provide a known input power); see
Section~2.1 in \citet{dempsey2012} and Section~5.3 in
\citet{holland2012} for more details. Finally, the conversion to
astronomical flux units from power involves a correction for
atmospheric extinction \citep[primarily using the JCMT Water Vapour
Monitor to track line-of-sight opacity variations, see Section~3
in][]{dempsey2012}, and the application of a flux conversion factor
(FCF) which is established from regular measurements of calibrators
such as Uranus \citep[Section~5 in][]{dempsey2012}.

In addition to uncorrelated white noise, and low-frequency (and often
correlated from bolometer to bolometer) $1/f$ drifts, bolometer power
spectra include a roll-off in their spectra at frequencies approaching
Nyquist, which is due to the anti-aliasing filter that forms part of
the 200\,Hz re-sampling stage. There are also line features which are
thought to be produced at frequencies far above the final 200\,Hz
sample rate, and are aliased to lower frequencies during the
multiplexed readout stage before they can be removed by the
anti-aliasing filter.

We note that the noise performance of \scuba\ has evolved over
time. An initial ``\scuba\ shared-risk observing'' period (S2SRO) took
place during February and March 2010, during which each of the 450 and
850\,\micron\ focal planes were populated with single subarrays, s4a
and s8d, respectively. In addition to having significantly fewer
available bolometers than the current fully-commissioned instrument
(and therefore a mapping speed reduced by a factor of $\sim$4),
instabilities in the fridge led to a large periodic signal with a
period of about 25\,s that was correlated amongst the detectors. After
upgrading and commissioning the instrument, a servo using newly added
thermometers in the focal planes has effectively mitigated this
problem \citep[Section~2.5 in][]{holland2012}. In addition, there were
improvements to the magnetic shielding (reducing magnetic field
pickup, as described in Section~\ref{sec:magpickup}), as well as more
effective removal of aliased noise sources (which has reduced both the
presence of line features and the mean white noise level).

In order to reduce the impact of low-frequency noise on the final
maps, \scuba\ scan strategies have been designed to provide: (i) good
cross-linking (visiting every point of the mapped area on several
different time scales, and at different scan angles); and (ii) minimal
accelerations to reduce turn-around overheads (which could be quite
large given the 600\,arcsec\,s$^{-1}$ maximum scan speed). For areas
larger than the array footprint a rectangular ``PONG'' pattern is
used, in which the boresight travels in approximately straight lines
and ``bounces'' off the edges at 45\,degree angles until the area is
uniformly filled in. It is also usually combined with a rotation
through a number of fixed position angles to create a ``rotating
PONG'' with even better cross-linking. For smaller areas (of order the
array footprint, or pont sources), in which the PONG turn-around
overheads would be large, a ``constant velocity daisy'' (CV Daisy) is
used.  Here the telescope moves in a circle, whose centre also slowly
traces out a small circle. For a more complete description of the
\scuba\ observing modes, see Section~5 in \citet{holland2012} and also
\citet{2010SPIE.7740E..66K}.

\begin{figure*}
\includegraphics[width=0.49\linewidth]{\imagefile{bolos_point_mix_s2sro}}
\includegraphics[width=0.49\linewidth]{\imagefile{bolos_point_mix}}
\caption{A comparison between single bolometer time-series in each of
  the 450 and 850\,\micron\ bands with the mixing chamber temperature
  and azimuth/elevation pointing offsets, before and after upgrading
  the instrument. The grey signals over-plotted in the top two panels
  show the residual time-series after removing the common-mode
  signals.  Left: data taken during the S2SRO period, observation 29
  on 20100313. There is a strong correlation between the bolometers
  and the roughly $\sim$25\,s oscillation in the fridge, but only a
  minor correlation with the telescope motion. The nearly flat
  common-mode subtracted signals show: (i) that most of the
  low-frequency signal is common to all of the bolometers; and (ii)
  the non-correlated, and predominantly white noise at 450\,\micron\
  is significantly larger than at 850\,\micron. Right: data taken with
  the fully-commissioned instrument, observation 38 on
  20111112. Unlike the S2SRO data, there is no strong signal produced
  by variations in the fridge temperature. The 450\,\micron\ data have
  minimal correlated low-frequency noise (as evidenced by the lack of
  a difference between the raw and common-mode subtracted signals),
  although the 850\,\micron\ data show a common signal correlated with
  the telescope motion (note that this scan has a significantly larger
  amplitude than the S2SRO data set), mostly likely caused by magnetic
  field pickup (see Section~\ref{sec:magpickup} and
  Fig.~\ref{fig:magpickup}).}
\label{fig:bolos_mix}
\end{figure*}

\begin{figure*}
\centering
\includegraphics[width=0.49\linewidth]{\imagefile{pspec_s2sro}}
\includegraphics[width=0.49\linewidth]{\imagefile{pspec}}
\caption{Bolometer power spectral densities for the same two data sets
  (before and after upgrades) used in Fig.~\ref{fig:bolos_mix}. The
  PSDs have been boxcar smoothed with a width of 0.1\,Hz to reduce the
  noise slightly and clarify some features. Four of the most sensitive
  bolometers have been selected at each wavelength, and the
  flat-fielded and step-corrected (but otherwise raw) PSDs are shown
  as coloured dotted lines (the blue signals are for the same time
  series as those shown in Fig.~\ref{fig:bolos_mix}).  The solid black
  lines are the PSDs of the common-mode signals at each wavelength,
  and the solid coloured lines show the PSDs of the bolometers once
  the common-mode is removed.  Finally, the dashed black lines show
  the spectral shape produced by a point source given the scan speeds
  for the two observations (120\,arcsec\,s$^{-1}$ in the left-hand
  plot, and 190\,arcsec\,s$^{-1}$ in the right-hand plot); for
  reference, the top horizontal axes shows the conversion from
  frequency to angular scale. Horizontal dotted lines at $2 \times
  10^5$ and $1 \times 10^5$\,pW\,Hz$^{-1}$ at 450 and 850\,\micron,
  respectively, are provided as a visual reference for the white-noise
  levels. In addition to $1/f$ and white noise components, and line
  features, all of the PSDs exhibit the gradual roll-off of the
  anti-aliasing filter just below the Nyquist frequency.  Left: for
  the S2SRO data there are broad line features in the PSDs at both
  wavelengths above $\sim$35\,Hz.  At lower frequencies, the bolometer
  signals exhibit clear $1/f$ knees at approximately 1 and 2\,Hz at
  450 and 850\,\micron. Common-mode subtraction removes most of the
  correlated fridge oscillation signal, lowering the $1/f$ knees to
  approximately 0.2 and 0.7\,Hz at 450 and 850\,\micron,
  respectively. Right: data from the fully-commissioned instrument
  tend to have lower $1/f$ knees, and fewer line features. The
  remaining low-frequency noise, now that the fridge oscillations have
  been removed, is considerably more independent from bolometer to
  bolometer than in the S2SRO data (note the larger spread in the
  dotted lines at low frequencies compared to the left-hand plot,
  particularly at 450\,\micron). Finally, this example shows that the
  white noise performance is in fact similar for the two subarrays
  (s4a and s8b) that were used both before and after the upgrades.}
\label{fig:pspec}
\end{figure*}


%-------------------------------------------------
\subsection{Typical bolometer signals}
\label{sec:bolosignal}
%-------------------------------------------------

In Fig.~\ref{fig:bolos_mix} we show sample time-series from single
bolometers in each of the 450 and 850\,\micron\ focal planes, as well
as variations in the mixing chamber temperature (though not located in
the focal plane itself, it is certainly correlated with the
temperature of the detectors), and the telescope pointing, for two
data sets, before and after the upgrades that followed S2SRO.

In the S2SRO data (left panel), observation 29 on 20100313, both
bolometers share significant long-timescale structure ($\gsim10$\,s)
that appears to be related to variations in the fridge base
temperature, although the similarity is clearly greater at
850\,\micron. In this particular case, the total power in the
fluctuations at 450\,\micron, are comparable to those at
850\,\micron. Such behaviour might be expected if there is a
comparable varying thermal load from the fridge at each wavelength
that dominates. We also note that there is no obvious strong
correlation between the low-frequency signal structure, at either
wavelength, with the telescope motion.

The low-frequency signal component of the S2SRO bolometer output is
also highly correlated amongst bolometers in the same subarray. We
have calculated a common-mode signal, $\mathbf{c}(t)$, as the average
time-series of all the working bolometers. We then fit the amplitude
of $\mathbf{c}(t)$ at each wavelength to the signals shown in
Fig.~\ref{fig:bolos_mix} and remove it, yielding the grey residual
signals. These residuals are quite flat, although still with
noticeable variations. The white noise is also apparent, and larger at
450\,\micron\ as one would expect from the larger backgrounds compared
to 850\,\micron.

Data from the fully commissioned instrument, observation 38 on
20111112, are shown in the right panel of
Fig.~\ref{fig:bolos_mix}. Having solved the fridge oscillation
problem, these data no longer exhibit a correlation with the mixing
chamber (however, note that variations are still seen in the mixing
chamber signal; such variations do not necessarily reflect changes in
the focal plane temperature). There is minimal correlated
low-frequency noise in these 450\,\micron\ data (note that there is
almost no difference once the common-mode has been subtracted). The
850\,\micron\ channel, however, exhibits a more significant signal
that is obviously correlated with the telescope motion. It is also
correlated amongst many of the detectors, and common-mode removal
corrects it to some extent. Part of the reason that this is seen here,
and not in the S2SRO data set shown, is that the amplitude of the scan
pattern is larger. Most of this ``scan-synchronous'' noise is
attributed to magnetic field pickup, as described in
Section~\ref{sec:magpickup}.

Next, in Fig.~\ref{fig:pspec} we produce power spectral density (PSD)
plots for four of the most sensitive bolometers from both focal
planes, using the same two data sets. To produce this figure, we
follow the convention that the PSD as a function of frequency,
$\mathbf{P}(f)$, is normalised such that the integral over frequency
gives the same variance as the time-series variance across the full
time-series. In other words, given a bolometer signal $\mathbf{b}(t)$,
%
\begin{equation}
\label{eq:psd}
\langle\mathbf{b}^2(t)\rangle = 2 \int_0^{f_\mathrm{N}} \mathbf{P}(f)
df ,
\end{equation}
%
where we only integrate over the positive frequencies up to
$f_\mathrm{N}$, the Nyquist frequency, and the factor of 2 accounts
for the (equal) power that appears at negative frequencies in the
discrete Fourier Transform. The units of the PSD written in this form
are pW$^2$\,Hz$^{-1}$.

The dotted coloured lines in Fig.~\ref{fig:pspec} show the PSDs for
raw, though flat-fielded and step-corrected (Section~\ref{sec:steps})
data. At each wavelength, and in both data sets, there are clear $1/f$
knees at frequencies ranging from roughly 1 to 2\,Hz, followed by a
predominantly white spectrum punctuated by line features at higher
frequencies, and finally a roll-off caused by the anti-aliasing filter
above $\gsim 70$\,Hz. The correlation between the low-frequency
components of the different bolometer signals is large. The solid
black lines in Fig.~\ref{fig:pspec} indicate the PSDs of the
common-modes $\mathbf{c}(t)$ at each wavelength, which reproduce much
of the low-frequency structure, as well as some of the
higher-frequency line features. The $\mathbf{c}(t)$ otherwise drop
substantially below the individual bolometer PSDs at high-frequency,
as expected if the bolometers are dominated by uncorrelated white
noise in that part of the spectrum. The common-mode signals are fit to
each bolometer time series and removed as in Fig.~\ref{fig:bolos_mix},
and the resulting PSDs are shown as solid coloured lines.  For
reference, the top horizontal axes have been converted to angular
scale assuming the scan speeds of each observation
(120\,arcsec\,s$^{-1}$ in the left-hand plot, and
190\,arcsec\,s$^{-1}$ in the right-hand plot). The power spectra of
unresolved point-sources are also shown as dashed lines in each band
(arbitrarily normalised), showing that the smallest features
resolvable by the telescope are only minimally affected by the excess
noise in the line features at these scan speeds (this may not be the
case at higher scan speeds).

In the S2SRO data (left-hand panel of Fig.~\ref{fig:pspec}), the
low-frequency noise is very correlated between the detectors (note the
tight scatter in the dotted coloured lines, and their resemblance to the
common-mode), due to it being dominated by the fridge
oscillations. Common-mode subtraction is very effective, reducing the
$1/f$ knees in both wavelengths by about a factor of 5.

Raw data from the fully-commissioned instrument (right-hand panel of
Fig.~\ref{fig:pspec}) generally have less significant $1/f$ noise as
compared with the S2SRO data, and it is considerably less correlated
amongst detectors (larger spread in the dotted coloured lines,
particularly noticeable in these 450\,\micron\ data), leading to a
less drastic improvement upon common-mode removal. However, since the
data are less dominated by low-frequency noise to begin with, the
signals generally require less agressive high-pass filtering to
produce maps, and therefore retain larger-scale structures than with
the S2SRO data. Furthermore, there are generally fewer line features
in the PSDs of bolometers in the fully-commissioned
instrument. Finally, these two data sets illustrate that the white
noise performance (NEFD) is in fact similar before and after the
upgrades for these two subarrays (s4a at 450\,micron, and s8b at
850\,\micron). The main improvements are a reduction in correlated and
line noise features mentioned above, the larger number of working
bolometers and field-of-view (approximately a factor of 4).


%-------------------------------------------------
\subsection{Magnetic field pickup}
\label{sec:magpickup}
%-------------------------------------------------

\begin{figure}
\centering
\includegraphics[width=\linewidth]{\imagefile{magpickup}}
\caption{Evidence for significant magnetic field pickup for
  observation 16 on 20100228 at 450\,\micron\ (s4a subarray).  The top
  panel shows two un-flatfielded (but mean-subtracted and step
  corrected) bolometer time-series from the same column, with a 200
  sample boxcar smooth (approximately 1\,s), illustrating that they
  are dominated by a similar signal with opposite signs. The second
  panel shows the dark squid signal for the column, also
  mean-subtracted and with the same boxcar smooth. The bottom panel
  shows the azimuthal and elevation offsets from the map centre (mean
  azimuth and elevation 171.9$^\circ$ and 68.0$^\circ$,
  respectively). Only the azimuthal signal is obviously correlated
  with the dark squids and bolometer signals, which suggests a
  magnetic field stationary with respect to the telescope dome as the
  source, since its direction with respect to the cryostat only
  changes with azimuthal motion.}
\label{fig:magpickup}
\end{figure}

An additional noise source that is significant in only a small subset
of the data (and more so during the S2SRO period) is magnetic field
pickup. Since the bolometer signals are ultimately detected through
the amplification of magnetic fields, any additional changing fields
within the instrument will also be detected.

Example data from the 450\,\micron\ subarray s4a where pickup appears
to be significant (observation 16 on 20100228) are shown in
Fig.~\ref{fig:magpickup}. The time-series for two bolometers in the
same column (not flatfielded) show that there is a strong signal with
a similar shape, but opposite signs. This behaviour is seen across the
entire array. The dark squid signal for the same columns exhibits a
similar shape and amplitude. Since the dark squid has no thermal
absorber or TES attached to it, this observed signal is not likely to
be optical or thermal in nature (although there can be some cross-talk
with the bolometers). Due to the fact that the sign of the gain in
each stage of SQUID amplification is random (although the combined
gain is constrained to be negative), and since magnetic field pickup
is only seen at the input to the second stage, the pickup can appear
with random signs for bolometers along a column, giving it a distinct
signature from other common signals that always appear with the same
sign.

The telescope pointing offsets for this approximately 0.5\,degree
diameter scan are also shown in Fig.~\ref{fig:magpickup}. Since the
phase of the azimuth offsets from the map centre in this scan pattern
slowly drifts with respect to the elevation offsets, it is clear that
the bolometer and dark squid signals are detecting a noise source that
is correlated only with the azimuthal motion and not the
elevation. This behaviour would be expected if if there were a strong
magnetic field fixed with respect to the telescope dome (i.e., the
earth's magnetic field). Since \scuba\ is mounted on a Nasmyth
platform, only azimuthal motion will change the direction of such a
field with respect to the cryostat. Tests have shown that, as in this
example, large scans in azimuth generically produce pickup. In
contrast, and as expected, changes in elevation results in pickup that
is approximately three orders-of-magnitude smaller.

%-------------------------------------------------
\subsection{Principal component analysis}
\label{sec:pca}
%-------------------------------------------------

A method that has been used to remove correlated noise as part of the
map-making procedure for Bolocam and AzTEC is Principal Component
Analysis \citep[PCA,][]{laurent2005,scott2008,perera2008}. Here we use
PCA to explore correlated signals in \scuba\ data.

The basic method is as follows: (i) a covariance matrix is built up
for all pairs $(i,j)$ of the $N$ bolometer time-series,
$\langle\mathbf{b}_i(t),\mathbf{b}_j(t)\rangle$; and (ii) a singular
value decomposition identifies a new set of statistically uncorrelated
eigenvectors, $\mathbf{\xi}_i(t)$ (i.e., whose covariance matrix is
diagonal), such that each of the bolometer time-series is a linear
combination of the eigenvectors, or components i.e., $\mathbf{b}_i(t)
= \bar{\mathbf{\xi}} \mathbf{\lambda}_i^\mathrm{T}$, where each row of
the matrix $\bar{\mathbf{\xi}}$ is an eigenvector, and
$\mathbf{\lambda}_i^\mathrm{T}$ is a column vector containing the
corresponding eigenvalues. The $\mathbf{\xi}_i(t)$ are normalised by
$[\sum_t \mathbf{\xi}_i^2(t)]^{1/2}$, such that the root mean square
(\rms) amplitude of each component in a given bolometer signal is
stored in the eigenvalues. In the earlier analyses mentioned, the
low-frequency noise is assumed to be encapsulated in those components
with the largest eigenvalues. Removing the projection of the time
series along those components then significantly reduces $1/f$ noise
while retaining most of the (higher-frequency) signal in point-like
sources

A novel feature of our \scuba\ analysis is that we can perform PCA
with 450 and 850\,\micron\ data simultaneously, potentially helping us
to differentiate thermal and optical noise signals (e.g., from the
atmosphere) that might appear in both wavelengths, from other noise
sources that are restricted to single subarrays (such as readout
noise). In Fig.~\ref{fig:pca} we show the results for a combined
analysis of the s4a and s8b subarrays for the first $\sim$100\,s of
the same observation that was used in the right-hand
(fully-commissioned) panels of Figs.~\ref{fig:bolos_mix} and
\ref{fig:pspec}. The 12 most significant components are shown (ranked
by the mean eigenvalues across all working bolometers in both
bands). The top and middle panels for each component show the
time-series and PSDs of the normalised eigenvectors. The bottom panels
are maps indicating the eigenvalues (amplitudes) of the component
across the focal plane (with the mean $\bar{\lambda}$ and and \rms,
$\lambda_\mathrm{rms}$, of the eigenvalues for all of the bolometers
calculated separately at each wavelength also shown).

The majority of the correlated signal at both wavelengths in
Fig.~\ref{fig:pca} is clearly produced by Components~1 and 2. In both
cases, the eigenvector time-series exhibit a roughly periodic signal
that resembles the scan pattern in the right-hand panel of
Fig.~\ref{fig:pca}. While these correlated signals also clearly appear
at both wavelengths, referring to the maps of eigenvalues, they are
considerably stronger at 850\,\micron\ (consistent with the visual
appearance of the bolometer signals in Fig.~\ref{fig:pca}). Also note
that the PSDs for these eigenvectors exhibit nearly a pure $1/f$
signature with almost no line features, or a white-noise plateau
(suggesting that these are high-\snr\ measurements of a purely
low-frequency drift).

\begin{figure*}
\centering
\includegraphics[width=\linewidth]{\imagefile{pca}}
\caption{The first twelve components from a principal component
  analysis, ranked by the mean eigenvalues, of combined 450 (s4a
  subarray) and 850\,\micron\ (s8b subarray) time-series bolometer
  data, for the same observation as that used in the right-hand panels
  of Figs.~\ref{fig:bolos_mix} and \ref{fig:pspec}. For each component
  the top plot shows the time-series of the normalised eigenvectors,
  the middle plot its PSD, and the bottom coloured panels the
  eigenvalues for the bolometers at each wavelength (the amplitudes of
  the eigenvectors in the time-series). For reference, both the mean,
  $\bar{\lambda}$, and \rms, $\lambda_\mathrm{rms}$, eigenvalues for
  the bolometers in each subarray are also shown. Most of the
  correlated $1/f$ signal is encapsulated in Components~1 and 2, and
  appears to be dominated by scan-synchronous noise (compare with the
  scan pattern in the right-hand panel of Fig.~\ref{fig:bolos_mix}).
  Components~3, 4, 7 and 11 are transient glitches that only appear in
  a small subset of the bolometers. Component~5 also appears in only a
  few bolometers, and is dominated by a narrow line at 57\,Hz which
  suggests that it is related to the 60\,Hz mains. Components~6 and
  8--10 contain a mixture of broad lines above $\sim$30\,Hz (which are
  probably aliased high-frequency noise), with some low-frequency
  drifts. Finally, the erratic appearance of the eigenvector for
  Component~12 is typical of a poorly-biased detector.}
\label{fig:pca}
\end{figure*}

Why are there two dominant components? First, the smoothly-varying
pattern in the eigenvalue maps (particularly evident at 850\,\micron)
suggests that the scan-synchronous noise (presumably magnetic field
pickup) has a different response across the focal plane; in this case
the PCA may have identified two orthogonal shapes that, when mixed in
different quantities, can reproduce most of the signal in each
bolometer. It is also likely that a varying atmospheric contribution
is also included in these eigenvectors. One would expect the
atmosphere to be more significant at 450 than at 850\,\micron,
although it is not obviously visible due to the strong presence of the
scan-synchronous noise.

Next, Components~3, 4, 7 and 11 in Fig.~\ref{fig:pca} have a common
character. In all cases, the PCA appears to have identified strong,
but transient glitches that appear in only one or several bolometers,
such as the two black bolometers near column 8, row 21 at
850\,\micron\ in Component~3. Such events are typically flagged and
removed during iterative de-spiking (Section~\ref{sec:ast}).

The most interesting feature of Component~5 in Fig.~\ref{fig:pca} is a
strong narrow line in the PSD at a frequency of about 57\,Hz
(suggesting the 60\,Hz mains as the culprit). The eigenvalue maps show
that it is only significant for several bolometers in the
850\,\micron\ subarray: at column 22 row 22; several bolometers at the
bottom of column 28; and more near the top-right of the subarray.

Components~6 and 8--10 in Fig.~\ref{fig:pca} have several generic
features in common. First, the PSDs exhibit moderate low-frequency
$1/f$ noise. Second, there are broad-line features at frequencies
above $\sim$30\,Hz which we believe to be aliased high-frequency
noise, in part due to the clear correlation patterns that appear along
columns in the eigenvalue maps at both wavelengths (e.g., columns 26
and 29 at 450\,\micron\ in Component~6). Finally, these components
have potentially unrelated features mixed into them, like sudden
offsets that are likely residuals due to imperfect step correction
(e.g., around $t\sim50$\,s in the eigenvector time-series for
Component~6; also see Section~\ref{sec:steps}).

Finally, Component~12 in Fig.~\ref{fig:pca} is significant in only a
handful of detectors in the 850\,\micron\ subarray (e.g., column 21
row 25). The time-series plot for the eigenvector exhibits a large
positive tail of sudden excursions, which gradually worsen over
time. Bolometers that exhibit this signal are probably poorly biased,
and are usually flagged due to elevated noise levels during
pre-processing (Section~\ref{sec:flagbadbol}).

This example illustrates some of the types of correlated signals and
patterns that PCA can identify in \scuba\ data. While there are
typically one or two strong signal components detected, in general,
the details can vary significantly from data set to data set, and the
lengths of the time-series analysed. The reason for this is probably
due to the fact that many of the noise sources are not stationary in
time (e.g., scan-synchronous noise which obviously depends on the scan
pattern, and the tuning of the subarrays). It should also be clear
from this example that while PCA offers some helpful insight into the
various sources of noise, it does not necessarily identify clean
patterns that can easily be modelled. For example, while there are
clearly correlation patterns along columns (Components~6 and 8--10,
particularly at 450\,\micron), as one might expect given the common
amplification chain, the intensities of these high-frequency signals
appear random in the eigenvalue maps, and a common-mode signal
estimated from the data for each column would not remove it. Finally,
we note that most of the high-frequency ($\gsim2$\,Hz) correlated
features that have been identified tend to be restricted to a small
number of detectors, and can often be identified and removed using
other (faster) means. What remains is predominantly low-frequency
noise. This conclusion is central to our data reduction strategy
described in the next section.

%------------------------------------------------------------------------------
\section{Production of maps}
\label{sec:algorithm}
%------------------------------------------------------------------------------

The approach taken by SMURF to reducing \scuba\ data is to model (and
remove) predominantly low-frequency noise sources that are correlated
amongst detectors, and to iterate this process along with estimates of
the map. Significant experimentation with different models for noise
sources during \scuba\ commissioning required a highly flexible
software framework, and a configurable user interface. To achieve this
goal, while minimizing development time, it was decided to build SMURF
as a Starlink package \citep{2009ASPC..411..418J}, which provides
access to a large suite of libraries (including a commanding and
messaging interface, astrometric coordinate conversions and generation
of standard WCS information, file formats etc.). Furthermore,
Starlink\footnote{\url{http://www.starlink.ac.uk}} is
open-source\footnote{\url{https://github.com/Starlink}} (distributed under the GNU General Public Licence v3), and already
used extensively at the Joint Astronomy Centre (host of the JCMT) for
many other systems \citep{jenness2011}, which helps with
interoperability. Though originally written in FORTRAN, many of the
core Starlink libraries are now ported to native C, or at least have a
C interface. It was therefore decided to develop SMURF in C as well
(rather than C++, for example, which would have required adding
further dependencies to Starlink), although we have taken an
object-oriented approach. For example, all of the data for a given
noise model are encapsulated in a C structure, and there is a standard
interface for all functions that handle models (member data and
functions for the class, respectively). In this way it is easy to
extend SMURF with new models. Parallelisation is incorporated in the
most time-consuming low-level routines using threads (e.g., performing
Fast Fourier Transforms, re-gridding the data), usually handling
either independent blocks of bolometers, or blocks of time, in each
thread, depending on the nature of the calculation. We have found that
for typical data sets the processing time scales well with the number
of central processing unit (CPU) cores, although beyond 8 the returns
are diminished due to tasks that cannot be parallelised (e.g., reading
the data from disk).

As we have seen in Sections~\ref{sec:bolosignal}--\ref{sec:pca},
\scuba\ data are dominated at low frequencies ($\lsim$2\,Hz) by highly
correlated signals. While it can be significantly reduced using simple
common-mode removal (subtracting the average signal from all
bolometers at each time step), the more complicated correlated
residuals at these low frequencies are difficult to model. Ultimately,
a very simple approach has been taken to handle both components, for
which the main steps in a typical reduction are shown in
Fig.~\ref{fig:dimm}.

\begin{figure}
\centering
\includegraphics[width=\linewidth]{\imagefile{dimm}}
\caption{Typical map-making algorithm. Raw data (stored in multiple
  files) are read and concatenated into continuous time-series, and
  flatfieded. Based on the scan speed of the telescope, the data are
  down-sampled to match the selected output map pixel size (usually 2
  and 4\,arcsec at 450 and 850\,\micron, respectively, to ensure
  adequate sampling of the \scuba\ PSFs). A cleaning stage repairs DC
  steps and spikes, and fills gaps to ensure continuity. Finally, the
  mean of each bolometer time-series is removed (higher-order
  polynomials may also be used). The iterative portion then begins
  (dashed box): estimating and removing the common-mode signal (the
  combined \model{COM} and \model{GAI} models); applying the
  extinction correction (\model{EXT}); high-pass filtering to remove
  residual independent low-frequency noise (\model{FLT}); estimating
  the map by re-gridding the data, and then removing its projection
  from the time-series (\model{AST}); and finally measuring the noise
  in the residual time-series (\model{NOI}). If the solution has
  converged, the map is written to disk. Otherwise any multiplicative
  factors that may have been applied to the data are inverted (i.e.,
  the extinction correction, \model{EXT}), and then the models for the
  low-frequency noise; the common-mode (\model{COM,GAI}) and high-pass
  filter (\model{FLT}). Each model is then re-estimated in turn until
  \model{AST}, at which point the previous estimate of the
  astronomical signal is added back into the data prior to its
  re-calculation.}
\label{fig:dimm}
\end{figure}

First, the raw data are passed through a pre-processing stage which
corrects some of the more significant glitches, applies flat-field
corrections etc. Next, the iterative process begins (dashed box). Most
of the low-frequency noise is removed using common-mode subtraction
(\model{COM,GAI}, Section~\ref{sec:comgai}). Then, the extinction
correction (\model{EXT}, Section~\ref{sec:ext}) is applied. At this
point the data resemble the grey bolometer traces in
Fig.~\ref{fig:bolos_mix}, with some residual baseline drifts still
visible. These drifts are removed using a high-pass filter
(\model{FLT}, Section~\ref{sec:flt}) implemented with Fast Fourier
Transforms (FFTs). The residual signal is then re-gridded to estimate
the map. Finally, the map is used to estimate and remove the
astronomical signal contamination in the bolometer data (\model{AST},
Section~\ref{sec:ast}), leaving a relatively clean data set in which
to measure the white noise level of each bolometer (\model{NOI},
Section~\ref{sec:noi}). Since the common-mode and filtering stages
will have introduced ringing in the map in the vicinity of bright
sources, the entire process is iterated. Each component in the dashed
box of Fig.~\ref{fig:dimm} is re-calculated in sequence. In this way,
the second time the common-mode is calculated, for example, most of
the bright astronomical sources in the data have already been removed
in the previous iteration when the map was estimated, reducing the
amount of ringing in the map once it is re-estimated. Note that the
extinction correction is a multiplicative factor, and must be inverted
prior to re-calculating any of the additive model components to
preserve the data amplitude. Additive model components from the
previous iteration are generically added back into the time-series
immediately prior to their re-calculation. However, for the special
case of the common-mode and high-pass filter stages, they are replaced
simultaneously at the start of the iteration to assist with
convergence (see Sections~\ref{sec:converge}).

In general, SMURF is highly configurable, including many options for
both the pre-processing stage and the iterative model components
(which models are used, what order they are calculated in, how they
are calculated).  In Sections~\ref{sec:dataprep} and
\ref{sec:components} we describe the data pre-processing steps and
iterative algorithm in detail. Section~\ref{sec:converge} explores
convergence tests and degeneracies between model components.


%-------------------------------------------------
\subsection{Data pre-processing}
\label{sec:dataprep}
%-------------------------------------------------

Prior to executing the iterative part of the algorithm, the data must
undergo several pre-processing steps. First, the data files are read
into memory and concatenated into continuous arrays (\scuba\ data are
broken up and written to disk every $\sim$30\,s regardless of the
observation length during data acquisition). As the data are loaded,
they are multiplied by the flatfield correction \citep[see Section~2.1
in][]{dempsey2012}. Next, a series of configurable data cleaning and
filtering procedures are applied; these include the removal of large
glitches that may hinder the iterative solution from converging, or
simply tasks that do not need to be iterated.

\subsubsection{Time-series down-sampling and map pixel size}
\label{sec:downsamp}

The highest useful frequency in the nominally 200\,Hz-sampled \scuba\
data is that which corresponds to the smallest angular scale that the
instrument is sensitive to. As mentioned earlier, the usual
rule-of-thumb for a Gaussian beam is to provide at least 3 samples for
each FWHM, or roughly 2.5\,arcsec for the 7.5\,arcsec 450\,\micron\
channel, and 5\,arcsec for the 14.5\,arcsec 850\,\micron\ channel. For
a typical scan speed of 300\,arcsec\,s$^{-1}$, the maximum useful
sample rate is therefore about 120\,Hz at 450\,\micron, and 40\,Hz at
850\,\micron. In order to save execution time and memory usage (both
of which scale linearly with data volume), it is clearly advantageous
to re-sample the data to these lower rates. In practice, the default
map pixel sizes are set to 2 and 4\,arcsec at 450 and 850\,\micron,
respectively (slightly over-sampled), and down-sampling occurs as the
data are loaded to match this spatial sampling given the slew speed.

The method used by SMURF is to average together multiple samples from
the original time-series, $x_i$ to estimate the lower-frequency output
time-series, $y_j$. In general there will not be an integer number of
samples from $x_i$ in each of the $y_j$, so fractional samples from
$x_i$ are used at the $y_j$ time boundaries. The algorithm is fast,
since each sample in $x_i$ need only be visited once. From a spectral
point of view, this boxcar average is equivalent to applying a
sinc-function low-pass filter. Such behaviour is desirable, since it
serves as an anti-aliasing filter; most non-white features above the
Nyquist frequency in $y_j$ that could contaminate the output data are
removed.

As an alternative, we also investigated an algorithm in which the FFT
of $x_i$ is simply truncated to the target sample rate before
transforming back to the time domain (i.e., applying a hard-edged
low-pass filter). In practice the noise performance was
indistinguishable, and required slightly longer execution time than using
the method above. Therefore this method is not used.

\subsubsection{Time-domain de-spiking}
\label{sec:timedespike}

Spikes of short duration and high amplitude are often seen in the time
series data. If not removed, they can cause ringing when filtering the
data. Two alternative approaches may be used to remove these
spikes. This section describes the detection and removal of spikes
within the time-series of each bolometer, and Section~\ref{sec:ast}
describes the iterative detection and removal of spikes as part of map
estimation. In practice, map-based de-spiking usually gives superior
results, and so time-domain de-spiking is switched off by default.

Each one-dimensional bolometer time-series is processed
independently. At each time slice, the median value of the current
bolometer is found in a box centred on the time slice, and containing
a specified number of time slices (typically 50). If the residual
between the time slice value and the median value is greater than some
specified multiple (typically 10) of the local noise level, the time
slice is flagged as a spike.

If the local noise level were estimated within the same box used to
determine the median value, a spike in the box would cause the local
noise level to be over-estimated severely. For this reason, the local
noise level is taken as the standard deviation of the values within a
neighbouring box on the ``down-stream'' side of the median box (that
is, the side that has already been checked for spikes). In other
words, the high end of the noise box is just below the low end of the
median filter box. This introduces a slight asymmetry in the noise,
but this should not matter unless the noise varies significantly on a
time scale shorter than the box size.

This simple algorithm is not very good at distinguishing between spikes
and bright point sources, and so the threshold for spike detection is
usually raised when making maps of bright point sources.

\subsubsection{Step correction}
\label{sec:steps}

\begin{figure*}
\centering
\includegraphics[width=\linewidth]{\imagefile{steps1}}
\caption{Examples of steps in time-series data. Steps occur with a wide
range of heights from the large steps shown on the left to the small
steps shown on the right. Large steps are often followed by a brief
``over-shoot'' as shown on the left. In both plots, the black curve is
the uncorrected time-series, and the red curve is the corrected time
series. Samples close to a step are omitted in the corrected time-series.
In the right hand plot, the blue curve is the uncorrected time-series for
a nearby bolometer. The similarity between the red and blue curves shows
that the step correction is performing well.
}
\label{fig:steps1}
\end{figure*}

Sudden steps can occur in the time-series data from each bolometer,
with the most likely cause being cosmic ray events \citep[see
Section~3.5.3 in][]{holland2012}. The black curves in
Fig.~\ref{fig:steps1} show examples of such steps in the time-series
for two bolometers. If not removed, these steps can cause severe
ringing when filtering, and visible streaks in the final map,
corresponding to the paths of individual bolometers over the sky.

Steps occur with a wide range of heights and shapes. The ratio of step
height to noise can vary from less than 10 to several hundred. Some
steps occur over a single sample, such as the step close to sample
5000 in the right-hand plot of Fig.~\ref{fig:steps1}, but others
happen more gradually, such as the step close to sample 5300. In
addition, a step can be preceded or followed by a short period of
instability, as is visible at the bottom of the step in the left hand
plot of Fig.~\ref{fig:steps1} (this is probably due to the response of
the \scuba\ anti-aliasing filter; the sudden large step occurs prior
to the 200\,Hz re-sampling). Further problems are caused by steps that
occur close together in time, such as the large downward step followed
by a smaller upward step close to sample 5000 in the right hand plot
of Fig.~\ref{fig:steps1}.

Detecting and correcting such a wide variety of steps reliably has
proved to be a challenge. In outline, the following stages are
involved in detecting steps in a single bolometer time-series:

\begin{enumerate}

\item median smooth the whole time-series;

\item find the gradient of the median smoothed time-series at each
sample;

\item smooth the gradient values to determine the local mean gradient
and subtract this local mean from the total gradient to get the
residual gradient;

\item find residual gradient values that exceed 25 times the local RMS
of the residual gradients;

\item group these high residual gradients into contiguous blocks of
samples;

\item merge blocks that are separated by less than 100 samples.
\end{enumerate}

The above process produces a list of candidate steps in each bolometer
time-series. Each candidate step is then verified, measured and corrected
using the following procedure:

\begin{enumerate}

\item the above process can misinterpret the edges of a bright source
as a step, so we ignore blocks that occur close to bright sources;

\item if the block passes the above test, a least squares linear fit
is performed to the median-filtered bolometer data just before the
block, and this fit is extrapolated to predict the data value expected
at the centre of the block on the basis of the preceding data;

\item a least squares linear fit is performed to the median-filtered
bolometer data just after the block, and this fit is extrapolated to
predict the data value expected at the centre of the block on the
basis of the following data;

\item the difference between these two expected data values is taken
as the step height;

\item the preceding three steps are repeated several times, each time
including a different selection of samples in the two least squares
fits, with the mean and standard deviation of the corresponding set of
step heights found;

\item if the mean step height is small compared to the standard deviation
of the step heights, or compared to the noise in the bolometer data, then
the step is ignored;

\item if the above checks are passed, all subsequent bolometer samples
are corrected by the mean step height;

\item bolometer samples within the duration of the step, and a few
samples on either side, are flagged as unusable.

\end{enumerate}

Once all steps have been corrected within a bolometer time-series, a
constant value is added to all samples in the time-series to restore its
original mean value.

The results of step correction are shown by the red curves in
Fig.~\ref{fig:steps1}. For comparison, the blue curve shows the
uncorrected time-series from a nearby bolometer that does not suffer from
steps. The agreement between the red and blue curves confirms that
the step correction algorithm is working satisfactorily.


\subsubsection{Gap filling / apodisation}
\label{sec:gaps}

SMURF uses FFTs of the bolometer data extensively for filtering.  Data
that have been flagged as bad for any reason (for instance, due to the
presence of spikes, steps, or unusual common-mode signal) need to be
excluded from this FFT. For this reason, each contiguous block of bad
data samples is filled with artificial data before taking the FFT. A
least squares linear fit is performed to the 50 samples preceding the
block, and a similar fit is performed to the 50 samples following the
block. These are used to estimate the expected values at the start and
end of the block of bad values. The bad values are then replaced by
linear interpolation between the expected start and end
values. Gaussian noise is added with a standard deviation equal to the
mean of the RMS residuals in the two fits.

In addition to replacing bad samples before the FFT, it is also
necessary to ensure that the data values at the start and end of the
time-series are similar. Since an FFT treats the data as a single
cycle in an infinitely repeating waveform, any large difference
between starting and ending values will effectively introduce sudden
steps at the start and end of each cycle, causing unwanted
oscillations (ringing) in the transform. Another consequence of the
cyclic nature of the FFT is that features at one end of the time
series can affect the filtered values at the other end of the time
series. Two methods are available to avoid these two problems:

\begin{enumerate}

\item Apodisation: a number of samples at the start and end of each
bolometer time-series are multiplied by a cosine function in order to
roll the data values off smoothly to zero. The default number of
samples modified at each end of the time-series is given by half the
ratio of the sampling frequency to the lowest retained frequency. In
addition, each end of the time-series is padded with double this
number of zeros. This method is illustrated in Fig.~\ref{fig:pad2}. It
is not used by default as it reduces the amount of data available for
the map, and can significantly hinder very short observations (e.g.,
of calibrators when focussing the telescope).

\begin{figure}
\centering
\includegraphics[width=\linewidth]{\imagefile{pad2}}
\caption{The black curve shows a bolometer time-series, padded
with zeros. The red curve shows the time-series after apodisation.}
\label{fig:pad2}
\end{figure}

\item Padding with artificial data: instead of padding with zeros,
each time-series is padded with artificial data that connects the two
ends of the data stream smoothly and includes Gaussian noise. No
apodisation is performed. The number of samples of padding at each end
is again equal to the ratio of the sampling frequency to the lowest
retained frequency.  This is illustrated in Fig.~\ref{fig:pad1}. See
\citet{stompor2002} for a thorough discussion of this procedure within
the context of CMB map-making. This method is used by default.

\begin{figure}
\centering
\includegraphics[width=\linewidth]{\imagefile{pad1}}
\caption{The black curve shows the same bolometer time-series as in
Fig.~\ref{fig:pad2}, again padded with zeros. The red curve shows the
time-series after padding with artificial data.}
\label{fig:pad1}
\end{figure}

\end{enumerate}

\subsubsection{Bolometer filtering}

Despite the ability of the map-maker to iteratively remove many noise
components, under some circumstances it may be desirable to filter the
data once during the pre-processing step. Three main filtering options
are available:

\begin{enumerate}

\item The most commonly-used pre-processing filter is polynomial
subtraction. A polynomial of the requested order is fit and removed
from each bolometer time-series. At a bare minimum, the mean is
removed from all of the bolometers (order 0) in all reductions
described in this paper.

\item All of the filters available as part of the iterative Fourier
Transform Filter (Section~\ref{sec:flt}) can also be applied once
during pre-processing.

\item As an alternative to the iterative map-making procedure,
cleaning by principal component analysis (PCA) is available as an
experimental pre-processing option (Section~\ref{sec:pca}). The most
significant components in the analysis are identified, and the
projection of each bolometer time-series along their eigenvectors are
removed. A single parameter specifies the threshold on the amplitude
of the eigenvalues to be removed, as a number of standard deviations
away from the mean value. Subarrays are cleaned independently, once,
for the full length of each continuous chunk of data. Given the
computational expense of this method, and initial tests which showed
little improvement over simple high-pass filtering, this method has
not yet been explored in detail with \scuba\ data. However, since
systematic effects seem to come and go, it is possible that PCA could
be useful for particular data sets.

\end{enumerate}

\subsubsection{Additional bad bolometer rejection}
\label{sec:flagbadbol}

Despite the cleaning operations described in the previous sections,
the data from a given bolometer may be unusable due to it being poorly
biased, having an incorrect flatfield correction applied, or having
some other particularly pathological noise contamination. Most of
these bolometers can be flagged simply by identifying outliers in the
distribution of bolometer white noise levels. By default, SMURF
measures the PSD of all bolometers between 2 and 10\,Hz, after all
other pre-processing steps have been run (identical to the measurement
in Section~\ref{sec:noi}). Despite having significant low-frequency
noise (and possibly bright astronomical source) contamination, this
higher-frequency portion of the PSD is generally quite
clean. Furthermore, even if there is contamination, the purpose of
this measurement is to identify outliers, rather than provide a
meaningful absolute measurement of a given bolometer's white noise
level. Both high and low (e.g., due to an incorrect and very small
flatfield correction being applied) outliers from the centre of the
logarithm (to reduce the impact of outliers) of the bolometer noise
distribution are flagged.

%-------------------------------------------------
\subsection{Iterative model calculation}
\label{sec:components}
%-------------------------------------------------

Once the data have been cleaned, and the worst data flagged, the
iterative solution for the map and contaminating noise signals begins.

First, we describe the generic model for \scuba\ data. We express the
digitised signal observed by the $i$th bolometer as a function of
time,
%
\begin{equation}
\mathbf{b}_i(t) = f_i[\mathbf{e}_i(t) \mathbf{a}_i(t) + \mathbf{n}_i(t)],
\label{eq:model}
\end{equation}
%
where $\mathbf{a}(t)$ is the time-varying signal produced by scanning
the telescope across astronomical sources, $\mathbf{e}(t)$ is the
time-varying extinction, which is a function of the telescope
elevation and atmospheric conditions, and $\mathbf{n}_i(t)$ represents
sources of noise. The two terms in square brackets, as written, have
units of power delivered to the detectors (pW). The scale factor $f_i$
converts this effective power to the digitised units recorded on disk
(DAC) -- the flatfield multiplied by a digitisation constant (applied
once during pre-processing) -- which in this formulation is assumed to
be constant in time.

We then express the noise, $\mathbf{n}_i(t)$, as the sum of several
components,
%
\begin{equation}
  \mathbf{n}_i(t) = \mathbf{n}^\mathrm{w}_i(t) +
  g_i\mathbf{n}^\mathrm{c}(t) + \mathbf{n}^\mathrm{f}_i(t),
\label{eq:noise}
\end{equation}
%
where $\mathbf{n}^\mathrm{w}_i(t)$ is uncorrelated white noise,
$\mathbf{n}^\mathrm{c}(t)$ is a correlated or common-mode signal (with
an optional scale factor $g_i$ for each bolometer), and
$\mathbf{n}^\mathrm{f}_i(t)$ is (predominantly low-frequency) noise in
excess of the white noise level, that is either uncorrelated from
bolometer-to-bolometer, or at least does not have a simple correlation
relationship that would lead to it being included in
$\mathbf{n}^\mathrm{c}(t)$.

During pre-processing, the map-maker divides by $f_i$ once. Then, in
each iteration, the map-maker divides by $f_i$, models and removes
$g_i\mathbf{n}^\mathrm{c}(t)$, divides by $\mathbf{e}_i(t)$, and
applies a high-pass filter to remove $\mathbf{n}^\mathrm{f}_i(t)$ from
the bolometer time-series. This procedure isolates the astronomical
signal and white noise, $\mathbf{a}_i(t) +
\mathbf{n}^\mathrm{w}_i(t)$, for estimating the map. Finally, once the
map is estimated it is projected into the time-domain and removed from
the bolometer time-series leaving $\mathbf{n}^\mathrm{w}_i(t)$, in
which the bolometer noise can be measured. This is the most typical
sequence; in practice the user can select an arbitrary order, although
the solution converges much faster when the strongest components are
estimated first, and then subtracted, leaving a cleaner signal for
subsequent model estimates.

After the first iteration, there will almost certainly be correlated
errors between the model estimates. For example, the presence of a
bright astronomical source will contaminate $\mathbf{n}^\mathrm{c}(t)$
(which is a simple average of all of the bolometer time-series at each
instant in time), leading to an over-subtraction, and in turn,
negative bowls in the map.

Subsequent iterations, however, diminish such problems. Each additive
model component is re-estimated in the same sequence, after first
adding the previous estimate back into the time-series (but
importantly, not the other components). In this case, much of the
astronomical signal will have been identified and removed in the
previous calculation of $\mathbf{a}_i(t)$, and therefore there will be
less contamination in $\mathbf{n}^\mathrm{c}(t)$, and less negative
bowling in the map. Any multiplicative models [e.g.,
$\mathbf{e}_i(t)$] are inverted immediately prior to the start of a
new iteration to preserve the units of the data. The iterative
solution will thus converge, barring degeneracies between model
components.

The full set of model components available, and the parameters that
control them, are described in the following
sections. Table~\ref{tab:components} shows the typical order in which
they are calculated. For a discussion on convergence tests and
degeneracies, see Section~\ref{sec:converge}.

\begin{table}
  \caption{Summary of the model components that can be fit to \scuba\
    time-series data with SMURF. Only the first group of models are
    typically fit to the data (\model{COM}--\model{NOI}) in the
    indicated order. The remaining models (\model{DKS}--\model{PLN})
    are usually omitted, although they are available as options.}
  \vspace{0.2cm}
  \centering
  \begin{tabular}{c|l}
    \hline
    Model & Description \\
    \hline
    \model{COM} & remove common-mode signal \\
    \model{GAI} & common-mode scaled to each bolometer \\
    \model{EXT} & extinction correction \\
    \model{FLT} & Fourier transform filter \\
    \model{AST} & map estimate of astronomical signal \\
    \model{NOI} & noise estimation \\
    \hline
    \model{DKS} & dark squid cleaning along columns \\
    \model{PLN} & 2-dimensional time-varying plane removal \\
    \model{SMO} & time-domain smoothing filter \\
    \model{TMP} & pointing as baseline template \\
    \hline
    \end{tabular}
  \label{tab:components}
\end{table}

\subsubsection{\model{COM,GAI}: common-mode estimation}
\label{sec:comgai}

Fig.~\ref{fig:com} shows the time-series from a selection of typical
bolometers.\footnote{The data have been flat-fielded and each time
series has been adjusted to a mean value of zero.} The similarity
between most bolometers is evident, and forms the common-mode signal
-- assumed to be a consequence of variations in the atmospheric
emission and fridge temperature. This common-mode usually dominates
the astronomical signal for all but the brightest sources, and swamps
faint extended structure.  The purpose of the \model{COM} and
\model{GAI} models is to remove this common-mode signal.

\begin{figure}
\centering
\includegraphics[width=\linewidth]{\imagefile{com}}
\caption{A selection of typical bolometer time-series after
flatfielding and removal of a constant baseline. It can be seen that
most bolometers exhibit a common time variation overlaid with other
features. Empty squares indicate the locations of broken bolometers.}
\label{fig:com}
\end{figure}

The \model{COM} model is the common-mode signal itself
[$\mathbf{n}^\mathrm{c}(t)$ in Eq.~\ref{eq:noise}]. It is a single
time-series that is estimated by finding the mean of all bolometers at
each time step. Bolometer values that have been flagged as unusable
are excluded from the mean.

Even after flat-fielding, bolometers may have slightly varying
sensitivities and so the amplitude of the common-mode variations will
also vary from bolometer to bolometer. Comparing each bolometer
time-series with the common-mode signal allows an estimate of the
relative bolometer sensitivity to be obtained. In practice, a least
squares linear fit is performed between the bolometer time-series and
the common-mode to determine a gain [$g_i$ in Eq.~\ref{eq:noise}] and
offset for each bolometer.  The gain and offset for each bolometer is
known as the \model{GAI} model. Each bolometer value can then
optionally be scaled and shifted using these values so that all
bolometers share a common (but as yet unknown) calibration. This
provides an alternative (or additional) flat-fielding strategy to that
described in Section~\ref{sec:bolos}.

An option exists to cater for time-varying sensitivities. In
principle, the gain of the bolometers should be constant in
time. However, there is evidence that there are slight variations, and
this option does tend to slightly improve the noise in maps.  If used,
the least squares fits described above are performed on short blocks
of contiguous time slices, providing multiple gain and offset values
for each bolometer (one pair for each block of time slices). The gain
and offset at any required time slice can then be found by
interpolation between these values.

It can be seen from Fig.~\ref{fig:com} that some bolometers depart
radically from the common-mode, indicating some problem with the
bolometer. Such bolometers are identified by calculating the Pearson
correlation coefficient between each bolometer time-series and the
common mode. Bolometers for which the correlation coefficient is below
a specified limit, or which have unusually high or low gains compared
to the other bolometers, are flagged as bad in order to omit them from
the final map. If the option described above for handling time-varying
sensitivities is used, then a correlation coefficient can be
determined for each individual block of time slices.  This allows
individual bad blocks to be rejected from a bolometer time-series,
rather than rejecting the whole bolometer. As a warning, rejecting
data based on outlier gain values can be misleading in cases where the
data are dominated by magnetic field pickup. For example, the
common-mode signal for the data in Fig.~\ref{fig:magpickup} would
resemble the azimuthal scan pattern, but clearly the gains will have
opposite signs (such that one or the other bolometer would be
erroneously flagged).

The common-mode value at each time slice is calculated as the
unweighted mean of the values of all bolometers that have not
previously been flagged as bad for some other reason. Bolometer values
that were flagged as bad simply because they were poorly correlated
with the common-mode on the previous iteration are, however, included
in the new common-mode estimate. If such samples are excluded, there
is a strong possibility of discontinuities appearing in the
\model{COM} model at block boundaries.  These in turn can lead to
ringing when filtering, and instabilities in the convergence process.

Any astronomical sources that are smaller than the array size will
contribute signal to some bolometers but not other bolometers, thus
biassing the simple mean used to estimate the common mode. However, on
each iteration of the map-making algorithm illustrated in
Fig.~\ref{fig:dimm}, a large fraction of the remaining astronomical
signal is extracted from the bolometer time-series and transferred to the
output map, resulting in subsequent estimates of the common-mode being
more accurate.

Any extended astronomical emission on a scale comparable to or larger
than the spatial extent of the area used to estimate the common-mode
will contribute a similar signal to all bolometers. Therefore such
extended emission is indistinguishable from the other sources of
common-mode signal (e.g., atmosphere variations) and will be removed
by the \model{COM} model. This places a limit on the spatial extent of
astronomical structure that can be recovered.

For this reason, the usual practice is to estimate a single
\model{COM} model by examining data from all four sub-arrays in each
waveband, since this allows spatial structure on the scale of the
whole focal plane to be recovered. However, sometimes there is
evidence that the common-mode differs from one array to another, and
so an option exists to estimate a separate \model{COM} model for each
individual subarray, with a consequent lowering in the scale of
spatial structure that can be recovered.

\subsubsection{\model{EXT}: extinction correction}
\label{sec:ext}

The extinction correction is a multiplicative factor that is normally
derived using the JCMT water vapour monitor (WVM), and is not
considered to be a free parameter in the solution [$\mathbf{e}_i(t)$
in Eq.~\ref{eq:model}]. However, it is applied as part of the
iterative solution, rather than a pre-processing step, since any small
errors will be amplified by the large low-frequency drifts in the raw
bolometer time-series. For example, if the bolometer drift is 1000
times greater than an astronomical source of interest, a 1 per cent
error in the flatfield will produce stripes of order 10 times the
astronomical signal amplitude in the final map! If \model{EXT} is
applied after the bulk of the low-frequency noise has been removed
(e.g., \model{COM,GAI}), then there is little potential for such small
errors to affect the final map.  For details on how it is calculated
see \citet{dempsey2012}. Note that its numerical value is calculated
only once, and simply applied as a scale factor in the iterative
solution. Unlike the additive model components, it is inverted at the
start of each iteration to preserve the amplitude of the data before
the re-calculation of other model components.

\subsubsection{\model{FLT}: Fourier transform filter}
\label{sec:flt}

This model takes the FFT of the bolometer time-series data, and can
apply both high- and low-pass filters, as well as notch filters, at
hard frequency edges specified by the user. Alternatively, the
frequency edges of the filters may be defined in terms of an angular
scale, but converted into a frequency through knowledge of the mean
telescope slew speed. The time-series are generally gap-filled
(Section~\ref{sec:gaps}) before the transform to avoid ringing
(primarily caused by wrap-around discontinuities at the ends of the
time-series). Finally, a whitening filter may also be applied in which
a simple form, $1/f^\alpha + \mathrm{constant}$, is fit to the power
spectrum of each bolometer, and then the bolometer FFT is multiplied
by the square root of its inverse (though normalised to preserve the
white noise level).  The signals that are removed from the time-series
by this process are stored in the \model{FLT} container
array. Typically this model is used purely as a high-pass filter to
remove most of the residual $1/f$ noise following common-mode removal
[$\mathbf{n}^\mathrm{f}_i(t)$ in Eq.~\ref{eq:noise}]. Low-pass
filtering is redundant for two reasons: (i) SMURF already low-pass
filters and re-samples to a lower sample rate, as described in
Section~\ref{sec:downsamp}; and (ii) the act of re-gridding the data
to produce map estimates effectively low-pass filters the data below a
frequency that corresponds to the inverse of the crossing time of a
single map pixel. Notch filters have not been proven to be useful with
\scuba\ data, particularly since line features tend to move around
(i.e., a dynamic line-detection system would have to be developed, and
care would have to be taken that astronomical sources are not
suppressed).

\subsubsection{\model{AST}: map estimation}
\label{sec:ast}

Map estimation is accomplished using a nearest-neighbour resampling of
the data onto a pre-defined map grid. For the $i$th map pixel,
$\mathbf{m}(x_i,y_i)$, the brightness is estimated as the weighted
average of the bolometer data samples $\mathbf{b}_j$ that land within
that pixel (from any bolometer or point in time),
%
\begin{equation}
  \mathbf{m}(x_i,y_i) = \frac{\sum_j \mathbf{w}_j \mathbf{b}_j }
                             { \sum_j \mathbf{w}_j } .
\end{equation}
%
For the initial iteration the weights $\mathbf{w}_j$ are set to 1, but
subsequently they are set to $1/\sigma_j^2$, the estimated inverse
variance expected from the bolometer white noise levels, as discussed
in Section~\ref{sec:noi}. This weighting scheme is sensible provided
that the bolometer data have no correlated (e.g., low-frequency)
noise.

In addition to the signal map, a variance map $\mathbf{v}(x_j,y_j)$ is
estimated. The default procedure is to estimate this quantity given
the scatter in the weighted samples. This is accomplished by dividing
the biased weighted sample variance by the number of samples that went
into the average (akin to the formula for standard error on the mean,
but accounting for weights),
%
\begin{equation}
\label{eq:varmap}
\mathbf{v}(x_j,y_j) = \frac{\sum_j \mathbf{w}_j
                            \sum_j \mathbf{w}_j \mathbf{b}_j -
                            \left( \sum_j \mathbf{w}_j \mathbf{b}_j \right)^2 }
                           { N \left( \sum_j \mathbf{w}_j \right)^2 },
\end{equation}
%
where $N_j$ is the total number of bolometer samples that land in the
pixel. As written, this algorithm is numerically unstable if the two
terms in the numerator are large, and of nearly the same value:
floating point precision errors can cause the difference to be
significantly incorrect. In practice we use the more numerically
stable ``weighted incremental algorithm'' that calculates incremental
differences, as described in \citet{west1979}.

We decided not to use an unbiased estimator (e.g., the extension of
the common $(1/N-1)\sum_j (\mathbf{b}_j - \bar{\mathbf{b}})^2$
estimator using weights), since in practice it would require
accumulating an additional array of values at every map pixel, and
only results in a small difference where there are less than $\sim10$
samples per pixel (a situation that is almost never encountered in a
\scuba\ map, except in the edge pixels).

Finally, once the map estimation is complete, the map is projected
into the time domain (the signal that would be produced in each
bolometer by the signal represented by the map, $\mathbf{a}_i(t)$ in
Eq.~\ref{eq:model}) and removed.

In addition to map estimation, the \model{AST} model can also be used
to perform map-based despiking of the time-series. Unlike the
time-domain despiker (Section~\ref{sec:timedespike}), this calculation
utilises the scatter in the population of samples used to estimate the
map pixel values (from different times and bolometers) to reject
outliers. This method is more robust against false-positive detections
of bright/compact astronomical sources, since transient features in
the time-series are unlikely to occur by chance whenever a bolometer
crosses a specific location on the sky, whereas real astronomical
sources are at fixed spatial locations.

The estimated variance, $\mathbf{v_p}(x_i,y_i)$, of the normalised
weighted samples that land in the $i$th map pixel is simply the biased
weighted sample variance (i.e., the variance map value multiplied by
the number of samples):
%
\begin{equation}
  \mathbf{v_p}(x_i,y_i) = N_i \mathbf{v}(x_i,y_i).
\end{equation}
%
In order to compare the weighted differences between the samples and
the map values, [$\mathbf{b}_j - \mathbf{m}(x_i,y_i)$] to
$\mathbf{v_p}$, they must be scaled appropriately. We define a
normalised difference, $\mathbf{d}_j$, in such a way that the variance
of this new variable gives the weighted sample variance of the
underlying data points:
%
\begin{eqnarray}
  \frac{\sum_j \mathbf{d}_j^2}{N} &=&
  \frac{\sum_j \mathbf{w}_j [\mathbf{b}_j - \mathbf{m}(x_i,y_i)]^2}
       {\sum_k \mathbf{w}_k} \\
   \Rightarrow \mathbf{d}_j^2 &=& \frac{N \mathbf{w}_j [\mathbf{b}_j -
       \mathbf{m}(x_i,y_i)]^2}{ \sum_k \mathbf{w}_k} .
\end{eqnarray}
%
The map-based despiker flags those $\mathbf{d}_j$ that are further
than some threshold number of standard deviations
$\sqrt{\mathbf{v_p}}$ away from zero, so that they are not used in
subsequent iterations.

A final option available as part of the \model{AST} model is to apply
constraints to the map to improve convergence, which presently include
setting user-specified or high-\snr\ regions to a value of zero for
all but the final iteration. See the discussion in
Section~\ref{sec:converge} and examples in
Sections~\ref{sec:examples}.

\subsubsection{\model{NOI}: noise estimation}
\label{sec:noi}

The primary purpose of the noise component, \model{NOI}, is to measure
the white noise levels of each bolometer [$\mathbf{n}^\mathrm{w}_i(t)$
in Eq.~\ref{eq:noise}]. The measurement generally occurs once all of
the other models have been fit and removed. This noise may then be
used to estimate weights for the data in subsequent iterations.

First, the bolometer PSDs are calculated as in Eq.~\ref{eq:psd}. An
average white noise level is then measured from 2 to 10\,Hz, a
relatively clean region of the PSD that tends to lie above the $1/f$
knee, but below the high-frequency line features for typical bolometer
data (Fig.~\ref{fig:pspec}). Taking this constant level for
$\mathbf{P}(f)$ we then calculate the expected variance of the
time-series (in approximately 200\,Hz samples) using
Eq.~\ref{eq:psd}. If the bolometer noise were produced purely by
uncorrelated sources (i.e., no other long time-scale drifts), with no
high-frequency line features, and without the attenuation at even
higher frequencies by the anti-aliasing filter, this is the
theoretical noise limit of the detectors. These measured variances are
stored, and then used in subsequent iterations to weight the data
points when calculating the map estimate (Section~\ref{sec:ast}). They
are also used for convergence tests (Section~\ref{sec:converge}).

Since noise estimation is usually calculated as the final step in the
iteration, the data at this stage have had most of the astronomical
and other large, low-frequency noise signals removed. For this reason,
\model{NOI} may optionally perform some cleaning options, such as the
DC step fixer (Section~\ref{sec:steps}) and spike detection
(Section~\ref{sec:timedespike}), which may work better with these
cleaner time-series.

Finally, it should be noted that the default procedure is to calculate
the white noise levels once within \model{NOI}, after the second
iteration. The reason for fixing these values is to prevent any
potential divergence in the weight estimates with iterations, and also
to provide a fixed reference for the convergence tests
(Section~\ref{sec:converge}). Note that the absolute values of the
noise, and therefore weights calculated by the model, are irrelevant
(only their relative values matter). The reason is that the final
noise in the map is measured empirically from the scatter of the
weighted data points that land in each pixel (Section~\ref{sec:ast}).

\subsubsection{Other experimental models}

Additional models exist as options, altough they are not generally
used: \model{DKS}, the use of dark squids as a template for removing
magnetic field pickup; \model{TMP} which uses the azimuth of the
telescope also as a template for magnetic field pickup; \model{SMO},
which provides a time-domain smoothing alternative to the FFT-based
filter model \model{FLT}; and \model{PLN}, which fits a plane to the
signal distribution across the focal plane at each time slice in an
attempt to remove possible coherent atmospheric sky-noise structure.

As described in Section~\ref{sec:magpickup}, wide scans can produce
significant magnetic field pickup that tracks the azimuthal motion of
the telescope. The \model{DKS} model uses the dark squid signals for
each column as a template that is simply scaled (gain and offset) to
each bolometer time-series before removal. Since not all columns have
working dark squids, another alternative model is \model{TMP} in which
the azimuth of the telescope itself is used as the template. In both
cases, while the models had some success at removing the magnetic
field pickup, they did not reduce the $1/f$ knee substantially,
necessitating the continued use of the \model{FLT} model to remove the
remaining low-frequency noise. Little or no difference in the final
maps could be seen with or without the inclusion of these models, and
so they are not generally used.

The \model{SMO} model uses a rolling mean or median boxcar filter to
calculate the low-frequency component of the bolometer signals, which
are then removed. In other words, this is an alternative to the
high-pass filtering for which \model{FLT} is generally used. The
primary reason for developing this model was to make it more robust
against ringing near the ends of the time-series, or residual spikes
(for which the median filtering is particularly useful). However, the
de-spiking and gap-filling algorithms that we have employed
(Section~\ref{sec:gaps}) successfully mitigate these problems, and the
\model{FLT} model is substantially faster.

Finally, we experimented with an alternative to \model{COM} for
removing correlated atmospheric noise. Rather than subtracting the
average signal at each time slice, \model{PLN} fits a plane to the
observed signal at each instant. We found that there was no obvious
improvement (either in terms of reducing the $1/f$ knee or reducing
the noise in final maps). This result is basically consistent with
those from earlier instruments, and suggests that the angular scale of
the emitting regions in the atmosphere are un-resolved at the \scuba\
focal plane. \citet{sayers2010}, in particular, examined several
different ways of modelling and removing the correlated atmospheric
noise from Bolocam data at 143 and 268\,GHz (also atop Mauna Kea),
finding only marginal improvements by fitting a plane, or even using
higher-order polynomials (see their figure~10, as well as the
discussion and references to earlier work in their section~4.3). In
\citet{aguirre2011} iterative PCA cleaning was used to remove the
atmosphere since the simpler methods mentioned left substantial
residuals in their data (similar to the \scuba\ data described here,
although PCA cleaning is prohibitively slow in our case).

%-------------------------------------------------
\subsection{Convergence tests and model degeneracies}
\label{sec:converge}
%-------------------------------------------------

The map-maker will halt if the convergence criteria have been
met. Presently there are two numerical tests that may be performed by
SMURF, or alternatively, a fixed number of iterations may be
specified. The first numerical test is an approximate check of the
change in reduced chi-squared, $\chi^2_\mathrm{r}$. The standard
deviations, $\sigma_i$, are measured for each of the residual
bolometer signals once the modelled signal components have been
removed. In the early iterations, the residual signals usually contain
both white noise, and other long-timescale features. However, once the
solution has converged, this signal should look approximately
white. Once the white noise, $\mathbf{n}^\mathrm{w}_i(t)$, has been
measured from the bolometer PSDs in \model{NOI}
(Section~\ref{sec:noi}), $\chi^2_\mathrm{r}$ is calculated as $(1/N)
\sum_i \sigma^2_i / \mathbf{n}^\mathrm{w}_i(t)$, where $i$ runs over
the $N$ bolometers. In other words, it is the average ratio between
the measured time-series bolometer variances and their white-noise
levels measured between 2 and 10\,Hz. Note that this expression does
not account for the degrees of freedom. Clearly there are a large
number of parameters in the model, which will ``fit-out'' some of the
uncorrelated white noise, and potentially bias this estimate of
$\chi^2_\mathrm{r}$ low. One approach would be to keep track of the
degrees of freedom, and therefore provide a correction, as in
\citet{kovacs2008}. However, note that we are only using this quantity
as a convergence test, and its value is not important in an absolute
sense. Furthermore, the reference white noise values are calculated
only once in SMURF (Section~\ref{sec:noi}), and the number of model
parameters are fixed, so the value of $\chi^2_\mathrm{r}$ should
decrease monotonically if the solution is behaving well. The
convergence criterion is met if the change in subsequent values of
$\chi^2_\mathrm{r}$ is smaller than the requsted threshikd, which is
set to $10^{-3}$ by default.

It was immediately found in the analysis of \scuba\ data that the
$\chi^2_\mathrm{r}$ test described above is not a sufficient stopping
criterion. The reason for this is that various components of the model
are, under normal circumstances, highly degenerate. One simple example
is the degeneracy between large-scale astronomical structures (larger
than the array footprint), and the common-mode rejection step
(\model{COM}), which will be illustrated in
Section~\ref{sec:point}. Since the flatness of the residual only
reflects the ability of the model to fit the shape of the bolometer
time-series, it does not necessarily correlate with convergence of the
most important model component: the map. Indeed, examination of the
map after large numbers of iterations (e.g., $>10$), shows the
presence of large-scale structures that are anti-correlated with
structures in other model components. For this reason, a second
map-based convergence test was added.

The map-based convergence statistic, $M_\mathrm{c}$, is the average
absolute change in the value of map pixels between subsequent
iterations, normalised by the map pixel uncertainties (square root of
Eq.~\ref{eq:varmap}), or
%
\begin{equation}
M^j_\mathrm{c} = \frac{1}{N} \sum_i \frac{| \mathbf{m}(x_j,y_j) -
  \mathbf{m}(x_{j-1},y_{j-1}) |} {\sqrt{\mathbf{v}(x_j,y_j)}} ,
\end{equation}
%
where $i$ runs over the $N$ map pixels, and $j$ enumerates the
iterations. The convergence criterion is met once this average change
is smaller than some threshold. Experimentally we found that a value
of 0.05 gives good results (on average, map pixels change by 5\% of
the estimated map RMS in subsequent iterations).  This provides good
correspondence with what we would choose ``by eye'', and letting the
solver run for many more iterations in several test cases yields
insignificant differences.

A major source of divergence is correlation between \model{COM} and
\model{FLT}. Since \model{FLT} usually consists of a high-pass filter
following the application of \model{COM}, \model{COM} is completely
free to grow any large-scale structure at frequencies below the chosen
filter edge. While such structure does not appear in the map (as it is
removed by \model{FLT}), we found that the solution could be made to
converge significantly faster by ``re-mixing'' \model{COM} and
\model{FLT} at the start of each iteration. In practice, at the start
of each iteration, the previous iterations of each model are both
added back into the residual immediately prior to re-calculating
\model{COM}. In this way, truly common-mode signals, even at
low-frequencies, do not leak into \model{FLT}.

The most serious cause of divergence is the degeneracy between
low-frequency signal that is removed, and large-scale structures in
the map. For example, by subtracting the common-mode signal, the map
is free to grow large-scale structures that exceed the angular scale
of the array footprint. Our approach as been to constrain regions of
the map devoid of sources to a value of zero for all but the final
iteration in the \model{AST} model (Section~\ref{sec:ast}). Such
regions are either user-defined in advance, or can be determined from
the data using a cut on \snr. This technique is explored considerably
in the examples from Sections~\ref{sec:point} and \ref{sec:extended}.

%------------------------------------------------------------------------------
\section{Examples}
\label{sec:examples}
%------------------------------------------------------------------------------

In this section three different common types of data are reduced: a
bright point source, Uranus (Section~\ref{sec:point}); a blind survey
of high-redshift galaxies in the Lockman Hole
(Section~\ref{sec:cosmo}); and a map of a bright extended star-forming
region in our Galaxy, M17 (Section~\ref{sec:extended}). In each case,
variations on the default algorithm from Section~\ref{sec:algorithm}
are described.

%-------------------------------------------------
\subsection{Known point source}
\label{sec:point}
%-------------------------------------------------

The accurate measurement of positions and brightnesses of known point
sources are necessary in real-time to establish telescope pointing
offsets and focus. They are also necessary to measure the FCF
(absolute calibration), and hence noise performance of the instrument
in astronomically-useful units.  In this example we produce a 450\,\micron\ map
of Uranus (observation 26 on 20111017), which is a nearly point-like
source for \scuba\ that is commonly used as a primary flux
calibrator. The CV daisy pattern was used, with a scan speed of 155
arcsec\,sec$^{-1}$. We perform several different reductions of the
data to illustrate the purpose of various model components and the
convergence properties of the solution (Fig.~\ref{fig:pointmaps}). In
all cases the maps are produced on a grid of azimuth (horizontal) and
elevation (vertical) offsets from the position of Uranus (the origin),
using 2\,arcsec pixels.

The first, simplest reduction of the data uses only the \model{COM}
model to estimate and remove the common-mode signal in order to
suppress low-frequency noise in the data. After \model{COM}, the
extinction correction is applied (\model{EXT}), and an initial map is
estimated using equal weighting for all of the detectors. This
estimate of \model{AST} is then removed from the data, and the noise
is measured in the residuals to estimate weights for the subsequent
and final iteration. The resulting map after these two iterations is
shown in Fig.~\ref{fig:pointmaps}a. While the \snr\ of Uranus is
clearly large ($\sim$250 as estimated by SMURF), enabling us to see
the faint sidelobes (circle and cross pattern), the map also has
obvious circular streaks and other large-scale ripples. The circular
streaks are due to the fact that \model{COM} does not account for all
of the low-frequency noise (see Fig.~\ref{fig:pspec}), and therefore
each bolometer leaves a trace of the circular scan pattern in the map,
as each of their baselines slowly drift independently. A significant
contribution to the larger-scale ripples in the map, however, can be
made by degeneracies in the map solution, as we discuss below.

To illustrate how large-scale ripples can form (and grow), the same
map solution is run for 100 iterations and shown in
Fig.~\ref{fig:pointmaps}b, now exhibiting a strong vertical
gradient. The degeneracy is easy to understand if the time-domain
behaviour of each model component is considered. The top panel of
Fig.~\ref{fig:degeneracy} shows the residual signals for a single
bolometer after 2 (black) and 100 (grey) iterations, which are nearly
identical, yet the change in the estimated \model{COM} (green) and
\model{AST} (red) signals between 2 and 100 iterations are
large. However, it is also clear that the estimated \model{COM} and
\model{AST} signals are complimentary. In other words, the large
change in the \model{AST} signal is cancelled by freedom in the
\model{COM} signal to grow with opposite sign. For comparison, the
bottom panel of Fig.~\ref{fig:degeneracy} shows the telescope pointing
for this section of data, and the shapes of the \model{AST} and
\model{COM} signals match the elevation component, which is aligned
with the gradient in Fig.~\ref{fig:pointmaps}b. Generically, the
calculation of \model{COM} will remove any information on angular
scales that are larger than the array footprint (outline shown in
Fig.~\ref{fig:pointmaps}b for reference), meaning that the map
solution is unconstrained on such large scales.

One improvement to remove residual low-frequency noise after
common-mode removal in this Uranus image, is to simply apply a
high-pass filter to the data. A third reduction of the data uses the
default map-making parameters, as described in
Section~\ref{sec:algorithm}, which adds the \model{FLT} model to
accomplish this task immediately prior to map estimation. We set the
filter edge based on an angular scale of 200\,arcsec, which, given the
scan speed of 155\,arcsec\,sec$^{-1}$, corresponds to a frequency of
0.78\,Hz. Now using the automatic map-based convergence test
(Section~\ref{sec:converge}), the resulting map converges after 10
iterations (Fig.~\ref{fig:pointmaps}c). Both the circular streaks and
the large-scale gradient are now removed, but they have been replaced
by an obvious, circularly-symmetric ringing pattern about Uranus. The
reason for this ringing is that the hard-edged high-pass filter in
frequency space is equivalent to a $\sinc$ function-like response in
map space. Since the scan pattern is fairly isotropic (scans at all
position angles), and there is a bright point-like source at the
centre, the result is an azimuthally-symmetric $\sinc$ function-like
pattern in the map.

This example illustrate the need for constraints in the map solution
in many situations. For calibrators (and other previously known
bright, compact sources), a good, simple prior is to constrain the map
to a value of zero away from the known locations of emission (inside
the \model{AST} model, Section~\ref{sec:ast}). In
Fig.~\ref{fig:pointmaps}d, a solution is produced in an identical
manner to Fig.~\ref{fig:pointmaps}c, but now setting the map
explicitly to zero beyond a radius of 60\,arcsec from the location of
Uranus (much larger than the FWHM of the main lobe), for all but the
final iteration. In this case, the map converges after 6 iterations,
and all of the previous artifacts have been effectively removed. Since
the map is now flat away from the source, and constrained to a value
of zero, it is appropriate for performing aperture photometry
directly, with no need for an additional reference annulus. This
approach to constrained map-making is similar to that employed for
poorly cross-linked scans of compact (though resolved) sources by
\citet{wiebe2009} using BLAST data.

\begin{figure*}
\centering
\includegraphics[width=\linewidth]{\imagefile{pointmaps}}
\caption{Multiple reductions of a 450\,\micron\ CV daisy scan of
  Uranus, all scaled between $-$0.002\,pW (white) and +0.002\,pW
  (black). (a) Reduction in which only common-mode subtraction is used
  to suppress low-frequency noise, and the reduction is forced to use
  2 iterations (after the first iteration an estimate of the source
  flux is removed, and the noise is measured in the residuals to
  obtain appropriate weighting for the second and final
  iteration). Circular streaks are caused by independent low-frequency
  noise that is not removed by the common-mode (Uranus peak
  0.15\,pW). (b) Same as (a) but using 50 iterations, illustrating the
  degeneracy between large-scale structure and the common-mode (the
  footprint of working bolometers is also shown for reference, Uranus
  peak 0.15\,pW). (c) Reduction in which high-pass filtering above
  0.775\,Hz (corresponding to a spatial scale of 200\,arcsec, as
  indicated) is applied after common-mode removal, but before the map
  estimate. The map-based convergence test is activated and the
  solution halts after 10 iterations, but leaving large-scale ringing
  due to the filter (Uranus peak 0.27\,pW). (d) Same as (c), but now
  the region of the map beyond the white dot-dashed circle is
  constrained to a value of zero until all but the final iteration.
  The map is now extremely flat, there is little attenuation of the
  source flux, and the diffraction pattern is clearly seen (Uranus
  peak 0.27\,pW).}
\label{fig:pointmaps}
\end{figure*}

\begin{figure}
\centering
\includegraphics[width=\linewidth]{\imagefile{degeneracy}}
\caption{Demonstration of the degeneracy between large-scale
  structures in the map and common-mode removal, corresponding to
  panels (a) and (b) in Fig.~\ref{fig:pointmaps}. The black and grey
  lines in the top panel show the residual signal for a single
  bolometer after 2 and 100 iterations, respectively (they are nearly
  identical; the black line lies beneath the grey line). The green and
  red lines show the difference between the \model{COM} and
  \model{AST} (the signal produced by the current map estimate for a
  given bolometer) model components for that bolometer between
  iterations 2 and 100, respectively. This shows that a strong signal
  has grown over time, and it has equal but opposite signs in the two
  model components, so that they cancel one another. The bottom panel
  shows the scan pattern of the telescope over the same period;
  clearly the \model{COM}/\model{AST} model degeneracy is correlated
  with the elevation offset, and referring to
  Fig.~\ref{fig:pointmaps}, panel (b), this corresponds to the strong
  vertical gradient that has appeared}.
\label{fig:degeneracy}
\end{figure}

%-------------------------------------------------
\subsection{Deep point source survey}
\label{sec:cosmo}
%-------------------------------------------------

\scuba\ surveys designed to detect extremely faint point-sources
(e.g., high-redshift star-forming galaxies, and features in debris
disks) are ideally limited by the white-noise performance of the
instrument. The approach described here for maximising the \snr\ of
point-sources involves three major steps: (i) generating a map that
removes most low-frequency noise sources with approximately linear
response, without prior knowledge of the location of sources; (ii)
whitening (essentially flattening) the map to make the noise
properties of the map easier to deal with; and (iii) detecting point
sources using a ``matched filter''. Note that variations on this
general procedure have been used extensively in the submm cosmology
community using previous instruments
\citep[e.g.,][]{scott2002,borys2003,laurent2005,coppin2006,scott2008,perera2008,devlin2009}.
In this section we reduce scans of the Lockman Hole taken during S2SRO
as a pilot project for the Cosmology Legacy Survey. It consists of
$\sim8.5$\,hours of data taken in 36 separate scans (average
15\,min. each) spread over February and March 2010, covering an area
of about 50\,arcmin$^2$. The full list of dates and observation
numbers includes: 20100218, 63, 64, 65, 70, 71, 72, 90, 91, 92, 97,
98, 99; 20100220, 111, 112, 113, 118, 128, 129, 130; 20100303, 61, 62,
64, 69, 70, 72, 73, 74, 75; 20100309, 87, 88, 90, 91; and 20100311,
59, 64, 65, 72. These observations represent about 80 per cent of the
total data taken as part of the project; the remaining observations
were dropped due to problems keeping the arrays properly tuned during
S2SRO, and were easily identified by their highly variable and erratic
bolometer time-series (which resulted in maps full of large streaks).


\begin{figure*}
\centering
\includegraphics[width=\linewidth]{\imagefile{lockman_maps}}
\caption{Reduction of a blank-field survey: the Lockman Hole. (a) Raw
  output of SMURF using high-pass filtering as a pre-processing step,
  followed by 4 iterations using only the \model{COM} model to remove
  residual low-frequency noise. White contours correspond to estimated
  noise levels of 1.25, 2.5 and 5.0 times the minimum noise at the
  centre of the map. (b) The whitened map using a filter based on the
  angular power spectrum of the jackknife map. (c) The whitened map
  cross-correlated with the whitened PSF to identify point sources
  [restricted to a lower-noise region, within the area denoted by the
  second contour of panel (a)]. The image shows the \snr\, with
  3.8-$\sigma$ peaks indicated by blue circles (radius 8\,arcsec). The
  orange ``$+$'' signs show the locations of 1.4\,GHz radio sources
  from \citet{owen2008} with $S\gsim15$\,$\mu$Jy. Of the 10 submm
  peaks, 9 are within this search radius of at least one radio
  source. (d) Jackknife map produced from the difference of two maps,
  using the even and odd scan numbers, respectively. Provided that all
  noise sources are statistically uncorrelated between the two halves
  of the data, the map is a plausible realisation of the noise without
  contamination from astronomical sources. (e) The jackknife map
  whitened using the same filter as that in panel (b). The whitened
  jackknife map cross-correlated with the same whitened PSF as in
  panel (c). Again, orange ``$+$'' and blue circles indicate radio
  sources and 3.8-$\sigma$ peaks, respectively. Unlike panel (c),
  there is only a single (apparently) significant peak, and it is not
  in the vicinity of a radio source.}
\label{fig:lockman_maps}
\end{figure*}

The first step, map generation, is different from that described in
Section~\ref{sec:point} in two key ways. Since the locations of
sources are unknown \emph{a priori}, a map constraint is not
employed. Large-scale diverging structures in the map must be
mitigated, and the method used in this example (the default processing
in SMURF) is to apply a high-pass filter to the data once, as a
pre-processing step. The iterative solution is then run using only
\model{COM}, \model{EXT}, \model{AST}, and \model{NOI}. In other
words, there is no information in the bolometer signals below some
cutoff frequency, and residual correlated high-frequency noise above
the cutoff is only removed through iterative common-mode
subtraction. Since the data are high-pass filtered prior to the
iterative solution, \model{GAI} (fitting an independent amplitude of
\model{COM} to each bolometer) has been de-activated, since there is
very little structure in the common-mode with which to fit an accurate
gain. The map is shown in Fig.~\ref{fig:lockman_maps}a. The map-maker
has been tested in two ways: (i) large numbers of iterations are used
to verify that the maps converge without the growth of large
structure; and (ii) adding synthetic sources to the time-series data
at a range of brightnesses verify that the map-maker response to them
is linear (i.e., the relative shape and amplitude compared to the
input source is independent of brightness). The response to a
synthetic point source (solid line) after map-making (dotted line) is
shown in Fig.~\ref{fig:lockman_psf}. Clearly the use of a high-pass
filter as a pre-processing step, and having no other map-constraints,
has the down-side of introducing sidelobes around the main
peak. However, the way this filter affects point-sources is both
well-known, and linear.

\begin{figure}
\centering
\includegraphics[width=\linewidth]{\imagefile{lockman_psf}}
\caption{Slice through the angular response to an ideal Gaussian point
  source (solid line) along the R.A. axis following map-making using
  the default blank-field processing with SMURF (dotted line), and
  upon application of the whitening filter (dashed line). The ``map
  filtered'' response is produced by adding the ideal Gaussian to the
  real data near the centre of the Lockman Hole map, and re-reducing
  the data. The resulting dotted line gives the expected shape of a
  source in panel (a) from Fig.~\ref{fig:lockman_maps}. Applying the
  whitening filter (reciprocal of the solid orange line in
  Fig.~\ref{fig:lockman_pspec}) to the dotted line gives the whitened
  line (dashed) , which is the expected shape of a source in panel (b)
  from Fig.~\ref{fig:lockman_maps}. Finally, cross-correlating the
  whitened map with this whitened PSF is an effective ``matched
  filter'' for identifying point-like sources, and this smoothed map
  is shown in panel (c) of Fig.~\ref{fig:lockman_maps}.}
\label{fig:lockman_psf}
\end{figure}

Even though the map looks quite flat, there is a mixture of faint
astronomical sources, and what is probably residual low-frequency
noise, causing faint patchiness visible to the naked eye. Since the
mixture of the two components is unknown, the first step is to
suppress the low-frequency noise, under the assumption that such
contaminants occur randomly in time, while astronomical sources are
(usually) constant.

First, the angular power spectrum of noise is estimated from a
``jackknife map'': maps are produced from two independent halves of
the total data set, and the jackknife signal in a map pixel,
$S_\mathrm{JK}$, and its variance, $\sigma^2_\mathrm{JK}$ are
estimated from the two input map fluxes, $S_1$, $S_2$, and the
corresponding variances, $\sigma^2_1$, and $\sigma^2_2$ as,
%
\begin{eqnarray}
S_\mathrm{JK} = \frac{S_1 - S_2}{2}, \\
\sigma^2_\mathrm{JK} = \frac{\sigma^2_1 + \sigma^2_2}{4}.
\end{eqnarray}
%
Provided that the noise in one half of the data is uncorrelated with
that from the other half, the signal in the jackknife map should
resemble noise drawn from the same parent distribution as that of the
real map. The astronomical signal, however, should be cleanly removed
(provided that there are no strong time-varying signals, and also
assuming that errors due to calibration between the two halves are
insignificant). The approach we have taken to minimize systematics is
to produce the two maps using odd and even scan numbers (i.e., each
map will contain a nearly uniformly-spaced sampling of data across the
full data set).

Since the \scuba\ scan strategy is usually isotropic (all position
angles scanned with roughly equal weights), we make the simplifying
assumption that the angular noise power spectrum is azimuthally
symmetric. For these data, there are no obvious anisotropic structures
in the 2-dimensional FFT. The radial (azimuthally-averaged) angular
power spectrum therefore encodes all of the useful information about
the noise properties. These power spectra for the raw output of SMURF,
and the jackknife map (transforming only the approximately uniform
region indicated by the square in Fig.~\ref{fig:lockman_maps}d in each
case) are shown by the dashed black, and solid orange lines in
Fig.~\ref{fig:lockman_pspec}, respectively. Both power spectra are
approximately flat at spatial frequencies $\gsim 0.06$\,arcsec$^{-1}$
(scales $\lsim16$\,arcsec), with the exception of a spike at
$\sim0.175$\,arcsec$^{-1}$ (a scale of $\sim5.7$\,arcsec). One
possibility for this feature is that it is related to the
inter-bolometer spacing in the focal plane (which happens to be this
size): small relative drifts in the baselines of adjacent bolometers
may produce faint parallel stripes in the map along the scan direction
(the superposition of many scans at different angles then results in
an isotropic noise pattern). It is not likely that this signal is due
to astronomical sources because it appears with a nearly equal
amplitude in both the real map and the jackknife.  At lower spatial
frequencies, both the real map and the jackknife power spectra grow
significantly, as a result of the more obvious large-scale patchiness
in Figs.~\ref{fig:lockman_maps}(a) and (d).

\begin{figure}
\centering
\includegraphics[width=\linewidth]{\imagefile{lockman_pspec}}
\caption{Radial (azimuthally-averaged) angular power spectral
  densities for the raw map output by SMURF (dashed black line), the
  jackknife (solid orange line), and whitened (solid black line) maps
  (panels (a), (d), and (b) from Fig.~\ref{fig:lockman_maps},
  respectively), considering only the square region indicated in
  Fig.~\ref{fig:lockman_maps}d. Since the raw map contains spatially
  correlated signals on large scales, both due to noise and
  astronomical signals (low-spatial frequencies), the jackknife map
  (difference of two approximately equal-length subsets of the data)
  is used to generate a plausible realisation of pure noise. Assuming
  that the noise properties are isotropic, a whitening filter is
  estimated from the reciprocal of the jackknife power spectrum with
  only a radial dependence. The power spectrum of the resulting signal
  map still has residual power on large scales, which is presumably
  due to astronomical sources.}
\label{fig:lockman_pspec}
\end{figure}

To suppress noise in the map, we construct a whitening filter from the
reciprocal of the jackknife angular power spectrum (orange line in
Fig.~\ref{fig:lockman_pspec}), normalised by the white-noise level
estimated from the RMS power at angular frequencies $>
0.1$\,arcsec$^{-1}$. The filter is applied by scaling the power
spectrum of the map by this function, and then transforming back into
real space. The whitened map is shown in Fig.~\ref{fig:lockman_maps}b,
and for comparison, the jackknife map has also been whitened in
Fig.~\ref{fig:lockman_maps}e. In both cases, the maps are visibly
flatter than the non-whitened cases.

The angular power spectrum of the whitened signal map is shown with a
solid line in Fig.~\ref{fig:lockman_pspec}. At low angular frequencies
($\lsim 0.5$\,arcsec$^{-1}$) there is significantly less power than
for the raw map. However, it also clearly has some power in excess of
the white noise level. In theory, this residual signal is produced by
astronomical sources, although its origin cannot be determined from
this plot alone; nor are astronomical sources readily visible in the
map (Fig.~\ref{fig:lockman_maps}b). To identify sources, we next apply
a matched filter to the whitened maps.

For blind, high-redshift surveys, individual sources are expected to
be un-resolved by the \scuba\ 7.5--14.5\,arcsec FWHM beams. Under this
assumption, cross-correlation between the map and the known PSF yields
the maximum-likelihood flux density of supposed point-sources centred
over every location in the resulting map \citep[an extremely
well-known result throughout astronomy, see][]{stetson1987}. Peak
identification in such smoothed maps have been used extensively in the
submillimetre community, as both an efficient source-detection and
photometric measurement strategy. In terms of the angular power
spectrum, a matched filter may also be thought of as an optimal
low-pass filter, that suppresses noise on scales smaller (and
therefore at higher angular frequencies) than the beam. For the case
at hand, the effective PSF for the whitened maps is given by the
dotted curve in Fig.~\ref{fig:lockman_psf}. Maps smoothed by this
shape are shown for the whitened signal, and jackknife maps in
Figs.~\ref{fig:lockman_maps}c and f, respectively. Note that these
images are plotted in \snr\ units, where the smooth noise maps have
been calculated by propagating the original noise maps output by SMURF
both through the whitening and matched filters (each of which are
linear operations).

Have real astronomical sources been detected using the matched filter?
For both the smoothed signal and jackknife maps, blue circles denote
3.8-$\sigma$ peaks. While not justified here, this threshold is fairly
typical for other ground-based submillimetre surveys in recent years
\citep[e.g.,][]{coppin2006,perera2008,2009ApJ...707.1201W} leading to
estimated false-identification rates of order $\sim$5\%, and is chosen
as a convenient reference. In the former, 10 peaks are found, whereas
there is only 1 in the latter. However, this test does not preclude
the possibility that some correlated noise made it into the jackknife
map, in which case the estimated \snr s are misleading.

One simple test of the calculated noise properties is to compare the
signal and jackknife \snr\ distributions with ideal Gaussians. The top
panel of Fig.~\ref{fig:lockman_hist} shows the whitened (but not
match-filtered) signal (blue) and jackknife (histograms), along with a
Gaussian (mean 0, standard deviation 1, and area normalised to the
number of map pixels) as a dashed line. In this case, it is clear that
the \snr\ distributions for both maps are nearly indistinguishable
from the theoretical distribution of white noise. This result shows us
that: (i) the whitening filter appears to have removed correlated
large-scale noise, since the jackknife map histogram is consistent
with white noise; and (ii) any potential astronomical signals are
small compared to the typical white noise in most map pixels
(unsurprising given the appearance of
Fig.~\ref{fig:lockman_maps}b). Next, we examine the \snr\ histograms
for maps processed with the matched filter in the bottom panel of
Fig.~\ref{fig:lockman_hist}. Again, the histogram of the jackknife
\snr\ data appears consistent with pure noise. However, the signal map
now deviates significantly from a Gaussian distribution, with a clear
positive tail (as one would expect for emitting sources). In fact,
integrating the positive tails beyond our 3.8-$\sigma$
source-detection threshold (vertical solid line) yields 229 map pixels
in the signal map, compared with 3 in the jackknife map (out of 80603
pixels in the entire region). Recall that given the small pixel size
of the map, several pixels generally contribute to each peak. This
procedure gives only a flavour of the analysis that is usually
required to produce a robust source lists. Additional tests, along
with a careful consideration of completeness and false-positive rates
usual require a series of Monte Carlo simulations that are beyond the
scope of this work. We direct the interested reader to a selection of
papers on the subject:
\citet{scott2002,coppin2006,perera2008,2009ApJ...707.1201W} and
\citet{chapin2011}.

\begin{figure}
\centering
\includegraphics[width=\linewidth]{\imagefile{lockman_hist}}
\caption{Histograms of \snr\ using pixels from the central region of
  the Lockman Hole (second contour) in
  Fig.~\ref{fig:lockman_maps}a. The top panel shows the histograms for
  pixels from the whitened signal and jackknife maps
  (Figs.~\ref{fig:lockman_maps}b,e), compared to a Gaussian
  distribution with mean zero, standard deviation one, and an area
  normalised to the total number of pixels -- the expected
  distribution for a map of spatially-uncorrelated Gaussian noise. The
  good agreement indicates that these maps are indeed dominated by
  white noise. The lower panel shows the results for matched-filtered
  signal and jackknife maps (Figs.~\ref{fig:lockman_maps}c,f). The
  filtered jackknife map distribution is still close to the
  expectation (Gaussian), but now the matched filter has picked out
  significant signal, leading to the large positive tail. The vertical
  solid line shows the chosen 3.8-$\sigma$ source-detection
  threshold.}
\label{fig:lockman_hist}
\end{figure}

As an additional external check, we have over-plotted orange ``$+$''
signs at the locations of 1.4\,GHz radio sources from \citet{owen2008}
with $S\gsim15$\,$\mu$Jy. Such radio catalogues have historically
proven invaluable for the precise identifications of high-redshift
submillimetre galaxies due to their relatively low surface densities
(compared with optical catalogues, for example), and a strong
correlation between the radio and submillimetre emission mechanisms
\citep[e.g.,][]{smail2000,pope2006,ivison2007,chapin2009b}. Taking a
representative search radius of 8\,arcsec from these studies with
similar \snr\ sources and source sizes (the same size as the blue
circles), 9 out of 10 peaks in the smoothed signal map have potential
radio counterparts, whereas the single peak in the smoothed jackknife
map does not lie near any radio source. Again, a proper
cross-identification analysis must inevitably include simulations to
establish completeness, false-positive rates, as well as the effects
of point source clustering and confusion. See the aforementioned
papers and references therein for examples.

\begin{figure*}
\centering
\includegraphics[width=\linewidth]{\imagefile{m17}}
\caption{An 850\,\micron\ rotating PONG map of M17. Intensity is
logarithmically scaled between $-$0.0003\,pw (white) and +0.01\,pW
(black). Iteration numbers are given in the corner of each
panel. Panels (a) and (b) show the results for a reduction using the
default parameters (the solution halted after reaching the default
map-based convergence criterion in 17 iterations). Panel (a) also
depicts the array footprint (position angle indicative of the start of
the observation), and a 300\,arcsec line shows the spatial scale
corresponding to the default \model{FLT} high-pass filter. Similar to
Fig.~\ref{fig:pointmaps}c, the lack of prior constraints on the map
leads to degeneracies, and therefore ripples on scales larger than the
array footprint (due to common-mode subtraction) and filter scale
(especially evident around the brightest peak at the centre of the
map). Panels (c) and (d) show the ``bright extended'' reduction, in
which a zero-mask is created iteratively from all of the pixels that
exceed an \snr\ of 5. While this region (red contour) only encompasses
the brightest peaks early in the solution, in the final iteration it
identifies most of the bright, extended emission, and significantly
helps with negative ringing.}
\label{fig:m17}
\end{figure*}

For future, significantly deeper \scuba\ maps, in which the RMS in a
PSF-smoothed map is dominated by point-source confusion, rather than
instrumental noise, a modified matched filter will offer improved
results. See Appendix~A in \citet{chapin2011}, which shows how to
include confusion (when known \emph{a priori}) explicitly as a noise
term in the calculation of such filters.

%-------------------------------------------------
\subsection{Bright extended emission}
\label{sec:extended}
%-------------------------------------------------

In this final example, we analyze a map of M17 which contains bright,
extended emission. The data are from observation 11 on 20110531 using
the 850\,\micron\ array. It is a rotating PONG scan covering a
diameter of 0.375\,deg, with a scan speed of 300\,arcsec\,s$^{-1}$,
and a transverse spacing of 180\,arcsec, taking 37.5\,min to complete.

The default reduction of these data is shown in the top panels of
Fig.~\ref{fig:m17}, after 2 iterations (the first map estimated after
the noise weights have been measured) and 17 iterations (when the map
has converged). The first panel also depicts the array footprint, and
the angular scale corresponding to the high-pass filter edge. Much
like the reduction of a point source without any prior constraints
(Fig.~\ref{fig:pointmaps}c), degeneracies on scales larger than the
array footprint and filter produce ripples in the map.

Unlike the case of a known point-source (Section~\ref{sec:point}), it
may not be possible for the observer to define, in advance, a mask of
regions containing blank sky. Indeed, for this map, much of the field
clearly contains extended structure. Furthermore, the goal of such
maps may be to detect previously unknown cool, dense regions of the
ISM that may not have appeared at other wavelengths (e.g., the first
optically-thick cloud-collapse stages of star-formation). While the
option does exist for the user to supply their own mask, a facility
has been added to SMURF to generate one automatically by flagging
pixels above some \snr\ threshold after each iteration as part of the
``bright extended'' configuration.

The results of this automatic masking are shown in the bottom panels
of Fig.~\ref{fig:m17}. After the second iteration, only narrow regions
around the brightest peaks are identified. However, as the solution
progresses, the negative bowls around the bright sources are slowly
reduced and the mask ``grows'' out from the brightest areas. By the
final iteration, most of the obvious structures in the data are
encompassed by the mask, negative bowling is significantly reduced,
and the brightest regions are more extended.


\begin{figure*}
\centering
\includegraphics[width=\linewidth]{\imagefile{m17_jk}}
\caption{Maps of M17 used to characterise the noise properties and
transfer function of SMURF (same intensity scale as
Fig.~\ref{fig:m17}). For each column, a different high-pass filter
edge scale was adopted (indicated in the top panels). First (top) row:
average of two halves of the data analysed independently, using the
default configuration. The data have had added to them synthetic
signal within a 600\,arcsec diameter circle (south of M17) created as
a realisation of noise from a $P(k) \propto k^{-3}$ angular power
spectrum multiplied by the PSF, subtracting the minimum to make it
positive, scaling it to a similar signal range as M17 itself, and
rolling-off the edges smoothly using half a period of a radial
(1+cosine)/2 function across 100\,arcsec beyond the edge of the
600\,arcsec region. Second row: jackknife maps produced from the
differences of the maps of each half of the data that went into the
first row. Third row: average of the two halves of the data using the
bright extended reduction. The blue and red contours indicate the
zero-masked regions for each half of the data (note for the
300\,arcsec case that the region about M17 fairly closely matches the
mask in Fig.~\ref{fig:m17}d). Clearly as the filter scale is
increased, due to the high \snr\ of the data, larger emission regions
are (correctly) detected and reproduced in the map. Fourth (bottom)
row: Jackknife maps for the bright extended reductions.}
\label{fig:m17_jk}
\end{figure*}

\begin{figure*}
\centering
\includegraphics[width=0.49\linewidth]{\imagefile{pspec_m17_default}}
\includegraphics[width=0.49\linewidth]{\imagefile{cor_pspec_m17_default}}
\caption{Angular PSDs for the region of the M17 map in
Fig.~\ref{fig:m17_jk} containing synthetic data, using the default
configuration. Left: raw PSDs for the input (noiseless) simulated data
(thick black line), the signal (average of each half) PSDs (thin solid
lines), and noise PSDs estimated from the jackknives (dashed
lines). Vertical dotted lines indicate the high-pass filter scales:
150\,arcsec (red); 300\,arcsec (orange); 600\,arcsec (green); and
900\,arcsec (blue). PSDs are also colour-coded by filter scale. Right:
since the input PSD is known, it is possible to measure the transfer
function of the map-maker as the ratio between the difference of the
output map signal PSDs and jackknife PSDs, and the input PSDs, giving
the thin coloured lines (linear vertical axis shown on right of
plot). The remaining lines are as in the left panel, but now corrected
by the transfer function. This plot shows that increasing the filter
scale improves the \snr\ at intermediates scales
($\sim$200--600\,arcsec), although the \snr\ worsens at smaller scales
($\lsim$200\,arcsec).}
\label{fig:m17_def_ps}
\end{figure*}

\begin{figure*}
\centering
\includegraphics[width=0.49\linewidth]{\imagefile{pspec_m17_bright_extended}}
\includegraphics[width=0.49\linewidth]{\imagefile{cor_pspec_m17_bright_extended}}
\caption{Lines have same meaning as in Fig.~\ref{fig:m17_def_ps},
except now using the bright extended configuration. This configuration
has resulted in an improved transfer function and \snr\ at large
angular scales. Also note that increasing the filter scale does not
have as large an impact on the small-scale noise as in the default
configuration, although it is still noticeably larger up to scales of
about 60\,arcsec.}
\label{fig:m17_be_ps}
\end{figure*}

While the reduction in the bottom panels of Fig.~\ref{fig:m17} is (at
least a cosmetic sense) superior to those in the top panels, it is
important to quantify both the noise properties of the maps, and the
response to real structures (the transfer function). We would also
like to know how each are affected by our choice of filter scale.
Similar to the previous section, we will use a jackknife test to
estimate the noise, as well as injecting known sources into the data
to observe how they are attenuated.

Maps are produced using the first and second continuous halves of the
data in Fig.~\ref{fig:m17_jk}. This is not an ideal situation, since
the noise properties may evolve with time (e.g., due to changing sky
conditions), leading to a biased estimate of the parent noise
distribution in the complete map from the jackknife. Also, since the
zero-masking depends on the \snr\ of the map, it will be restricted to
regions approximately $\sqrt{2}$ shallower in these maps than for the
full data set. Finally, the cross-linking (positional angles sampled)
is similar, though not identical in the two halves. Ideally one would
have many full maps, as in the case of the Lockman Hole data in the
previous section, from which alternating maps could be combined.

Since our goal in this section is to measure the response of the
map-maker to extended structures, we inject a simulated signal with
power at a range of scales into a relatively empty region of the map.
It is created by drawing a realisation of noise within an 800\,arcsec
on-a-side box, with an angular power spectrum $P(k) \propto k^{-3}$,
which is appropriate for Galactic cirrus clouds
\citep[e.g.,][]{gautier1992}. It is then filtered again with a
14.5\,arcsec Gaussian to model the effect of the \scuba\ optical
response. The RMS of these fluctuations is then normalised to 0.002 pW
so that they are comparable to the dynamic range of M17 itself. The
minimum is then subtracted so that the signal is positive. Finally,
half a period of a (1+cosine)/2 function is used to roll-off the
signal to zero between radii of 300 and 400 arcsec.

The first row of Fig.~\ref{fig:m17_jk} shows the total signal image
averaging the maps made of each independent half of the data, for the
default configuration (inverse-variance weighting has been used). The
columns show reductions using 150, 300, 600, and 900\,arcsec filter
edges. For reference, the largest scale that is completely inscribed
by the array footprint is about 400\,arcsec, and the diagonal of the
array is about 600\,arcsec (meaning that scales beyond 600\,arcsec are
completely unconstrained when common-mode rejection is used).  The
synthetic data are clearly seen as the circular region south of
M17. As the filter scale is increased, the size of the ripples
increases accordingly. While larger structures do seem to appear,
negative bowls are a major problem without any other map
constraints.The second row of Fig.~\ref{fig:m17_jk} shows the
jackknife maps. The astronomical emission is almost perfectly removed,
except for a slight impression of M17 near the centre of the map, that
is probably due to some combination of detector gain and pointing
drifts. Otherwise the jackknife appears to be a plausible realisation
of noise from the same parent distribution as for the averaged maps.

The third and fourth rows in Fig.~\ref{fig:m17_jk} repeat this
exercise using the bright extended configuration, in which the \snr\
threshold of 5 is again used to constrain the map. As the filter scale
is increased, more of the extended structure in M17 is reproduced in
the map, as evidenced by the blue and red contours (masks generated
from the first and second halves of the data, respectively). The
masking does a generally good job of suppressing the largest-scale
ripples that are produced by the default reduction. However, the noise
away from regions of bright emission does increase noticeably (mottled
appearance). With filter scales $\gsim600$\,arcsec, virtually the
entire region of synthetic emission is correctly identified by the \snr\
mask and allowed to vary freely in the solution.

Next, we analyse the angular power spectral densities (PSDs) of the
maps to understand the signal and noise properties of the map-maker in
the region of synthetic sources, as a function of filter scale. The
left panel of Fig.~\ref{fig:m17_def_ps} shows the raw PSDs for the
input synthetic signal (thick black line), the output map signals
(thin solid lines), and the jackknife maps (dashed lines). Colours
encode the filter scales used: 150\,arcsec (red); 300\,arcsec
(orange); 600\,arcsec (green); and 900\,arcsec (blue). Note that, with
the exception of the synthetic data, we only plot the PSDs down to the
second-lowest spatial frequency bin of
$1.875\times10^{-3}$\,arcsec$^{-1}$, or 533\,arcsec, due to the fact
that the smallest is very poorly sampled, and therefore noisy. As the
filter edge is increased, more power is measured in the map PSDs at
larger scales. However, much of this power is clearly produced by
noise which appears in the jackknife PSDs. We estimate the map-maker
transfer function as the ratio between the portion of the signal PSDs
not produced by noise, to the input PSD, or
%
\begin{equation}
G(k) = \frac{P_\mathrm{M}(k) - P_\mathrm{JK}(k)}{P_\mathrm{S}(k)},
\end{equation}
%
where $k$ is the spatial frequency, and the subscripts ``M'', ``JK'',
and ``S'' refer to the signal map, jackknife map, and synthetic map,
respectively.

The transfer functions $G(k)$ are plotted as thin solid lines in the
right panel of Fig.~\ref{fig:m17_def_ps}. This formula produces
extremely noisy values at large frequencies (small scales), and we
therefore set it to a value of 1 above 0.015\,arcsec$^{-1}$, as well
as any point in the curve where $P_\mathrm{M}(k)$ exceeds
$P_\mathrm{S}(k)$ (i.e., $G(k)$ is assumed to be $\le$1). As expected,
the larger the scale of the filter, the lower the spatial frequency at
which the map transfer function falls. We then correct the map and
jackknife PSDs by dividing by $G(k)$ to produce the thick solid, and
dashed lines in the right panel of Fig.~\ref{fig:m17_def_ps}. This
shows us that, even though the raw noise in the left panel is lower at
small scales when a smaller-scale filter is used (e.g., the red dashed
line), once we correct for the transfer function, we actually win in a
\snr\ sense at large scales using the larger-scale filter (the
corrected noise is lower), as would be expected.

These tests are then repeated using the bright extended reduction, in
Fig.~\ref{fig:m17_be_ps}. The most obvious improvement with this
reduction over the default reduction is that the transfer functions
fall more slowly at large angular scales, accompanied by a slower
increase in noise; in other words, there is greatly improved \snr\ at
large angular scales (an obvious conclusion given the appearence of
the maps in Fig.~\ref{fig:m17_jk}). In fact, using the 900\,arcsec
filter edge, the map response is still above 80 per cent right out to
the largest scale accurately measured in the PSDs, 533\,arcsec, which
is about the largest scale that should be recoverable, given the size
of the array footprint and the fact that we use common-mode rejection.
Another interesting feature of these reductions is that the increase
in small-scale noise as the filter edge is increased is not as drastic
as in the default reduction, and the \snr\ improves at scales
$\gsim$60\,arcsec.

One case in which the \snr\ is worse using the bright extended
reduction is when using a 150\,arcsec filter. In this case the noise
is considerably larger in the bright extended reduction, as evidenced
by the ``kink'' near 150\,arcsec. Referring to the mask contours in
the left panel of the third row, it is clear that the map-maker has
failed to identify much of the bright, extended emission in the region
of the synthetic source. Each area that is not within the contours is
constrained to zero throughout the solution, therefore suppressing
power (and lowering the transfer function), and subsequently reducing
the \snr\ of the final result. This measurement serves as a warning:
the map-maker response is non-linear when using \snr\ masking. Harsh
filtering can provide misleading results, as in this example. Maps of
faint extended emission will also suffer considerably, as the
structures of interest may not clear the \snr\ threshold for the mask.

Note that alternatives to the zero-masking approach do exist for other
iterative map-makers. For example, \citet{kovacs2008} typically
restricts the solution to a small fixed number of iterations
($\lsim$10), so that there is probably little chance that such
structures have the chance to grow. Experimentation with the model
order is also advocated to gain an impression of the convergence
properties. This approach is perhaps more relevant to their SHARC-2
data for which a more complicated model is developed; the degeneracies
we have observed for \scuba\ maps using our more brute-force approach
with a high-pass filter tend to look similar regardless of the order
(only the convergence time is affected).  For Bolocam data of the
Galactic Plane, \citet{aguirre2011} used a maximum-entropy filtering
step to suppress large-scales. Regardless of the method used,
simulations are always required to establish the transfer function and
noise properties as a function of angular scale.


%------------------------------------------------------------------------------
\section{Conclusions}
\label{sec:conclusions}
%------------------------------------------------------------------------------

This paper has described the Submillimetre User Reduction Facility
(SMURF), which was designed to produce maps from the rapidly sampled
$\sim10^4$ bolometers of the \scuba\ instrument. While similar to
other algorithms in the literature that iteratively model and remove
correlated noise components from the data, successively improving the
map \citep[e.g.,][]{kovacs2008,siringo2009,aguirre2011}, we have shown
that a relatively simpler approach provides good results for \scuba\
data, with reasonable computational requirements. This conclusion is
encouraging for the development of future large-format bolometer
arrays, for which our specific approach should be useful when even
larger volumes of data are involved.

A major obstacle to making maps of \scuba\ data is low-frequency
correlated noise (probably a mixture of atmospheric signals and
magnetic field pickup), which occurs at predominantly $\lsim$2\,Hz.
Much of this signal can be removed using common-mode rejection,
although principal component analysis (PCA) identifies a number of
other less-significant correlated noise patterns at low
frequencies. Unfortunately these patterns are difficult to model, and
their character varies from observation to observation. One could use
the PCA directly to remove them, but this technique is prohibitively
slow for existing high-end desktop computers, given the volume of
\scuba\ data. However, we have demonstrated that high-pass filtering
can remove the bulk of these residual correlated noise components
using significantly more efficient Fast Fourier Transforms (FFTs). The
basic algorithm therefore consists of iterative common-mode rejection
and high-pass filtering along with estimates of the map, which allows
us to approach the white-noise limit of the instrument.

We have found that the iterative solution tends to diverge on large
angular scales due to the degeneracy between the map, and the
low-frequency signal components that are removed. A simple strategy of
constraining empty regions of the map to zero (using either a
user-supplied mask for known sources, or an iterative determination of
signal above some \snr\ threshold) provides good constraints for both
compact objects, and bright/extended structures. Particularly in the
latter case, using a combination of synthetic sources and an empirical
measurement of the map noise from jackknife tests (differences of
independent portions of the data), we have demonstrated that we can
effectively recover angular scales up to the order of the array
footprint (approximately 5\,arcmin).

For maps of faint point sources, a single (non-iterative) high-pass
filter at the start of the reduction produces maps that are nearly
white-noise limited. Residual large-scale noise can be removed with a
whitening filter (also established from jackknife estimates of the
noise), and sources detected using a matched-filter (smoothing with
the effective filtered point spread function).

The iterative solution is stopped once convergence in the $\chi^2$
statistic, and/or variations in the map itself is achieved. This
enables SMURF to run in a pipeline setting without user interaction
for a wide variety of observations. Furthermore, the execution times
are shorter than the observation lengths, and memory requirements for
even the longest \scuba\ observations are within the capabilities of
single, high-end desktop computers. SMURF can therefore provide real-time
feedback at the telescope to observers, and it also produces nearly
science-grade data products for the JCMT Science Archive hosted by the
Canadian Astronomy Data Centre.

%SMURF is highly configurable, making it easy to vary the order
%and types of models that are fit to the data. It is also easy to add
%new models to the existing infrastructure.

One regime in which SMURF does not presently perform well is in maps
of faint extended structures, since the zero-masking technique we have
adopted cannot be used. Since SMURF is both highly configurable and
extensible, it may be possible to develop an improved data model
and/or map constraint to assist in these situations, as more
experience with the instrument is gained. However, provided sufficient
computing power is available, the best solution in the long-term will
be a maximum-likelihood algorithm, such as SANEPIC
\citep{patanchon2008}. Even in this case, the existing iterative
solution from SMURF will probably be used as an initial step, since it
can quickly clean the bolometer time-series, as well as perform
map-based despiking (a necessarily iterative procedure).

%------------------------------------------------------------------------------
\section{Acknowledgements}
%------------------------------------------------------------------------------

The James Clerk Maxwell Telescope is operated by the Joint Astronomy
Centre on behalf of the Science and Technology Facilities Council of
the United Kingdom, the Netherlands Organisation for Scientific
Research, and the National Research Council of Canada. Additional
funds for the contruction of \scuba\ were provided by the Canada
Foundation for Innovation.  This research was supported in part by the
Natural Sciences and Engineering Research Council of Canada.  EC
thanks CANARIE/CANFAR for additional funding.  The authors thank the
members of the SCUBA-2 commissioning team for testing the map-maker
and reporting anomalies; in particular Antonio Chrysostomou and Jessica
Dempsey.  We also thank Mark Halpern, Matthew
Hasselfield, and Gaelen Marsden for many useful discussions, and
observers who provided helpful feedback; especially David Nutter and
Todd McKenzie.  We acknowledge the contributions of Dennis Kelly, Alex
van Engelen and Jennifer Balfour for early investigations related to
SMURF; and Mark Thompson, Craig Walther and S\'{e}verin Gaudet for
being on the Critical Design Review panel. Finally, we thank Per
Friberg and Gary Davis for their helpful comments on the manuscript.


%------------------------------------------------------------------------------
\bibliographystyle{mn2e}
\bibliography{mn-jour,refs}
%------------------------------------------------------------------------------


\end{document}
